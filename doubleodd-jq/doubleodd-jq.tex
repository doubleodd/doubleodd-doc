\documentclass{llncs}
  
\usepackage[T1]{fontenc}
\usepackage[utf8x]{inputenc}
\usepackage[english]{babel}
\usepackage{lmodern}
\usepackage{mathtools}
%\usepackage{fullpage}
\usepackage{graphicx}
\usepackage{xspace}
\usepackage{tabularx}
\usepackage[lf]{ebgaramond}
\usepackage{biolinum}
\usepackage[cmintegrals,cmbraces]{newtxmath}
\usepackage{ebgaramond-maths}
\usepackage{multicol}
\usepackage{sectsty}
\usepackage[noend]{algpseudocode}
\usepackage{algorithm}
\usepackage{algorithmicx}
\usepackage[bottom]{footmisc}
\usepackage{caption}
%\captionsetup{font=small}
\captionsetup{labelfont={sf,bf}}
\usepackage{ragged2e}
\usepackage{xcolor}

% llncs removes generation of PDF bookmarks, we must add them back
\usepackage{etoolbox}
\makeatletter
\let\llncs@addcontentsline\addcontentsline
\patchcmd{\maketitle}{\addcontentsline}{\llncs@addcontentsline}{}{}
\patchcmd{\maketitle}{\addcontentsline}{\llncs@addcontentsline}{}{}
\patchcmd{\maketitle}{\addcontentsline}{\llncs@addcontentsline}{}{}
\setcounter{tocdepth}{3}
\makeatother
\PassOptionsToPackage{hyphens}{url}
\usepackage[pdftex,bookmarks=true]{hyperref}
\hypersetup{
    bookmarksopen=true,
    bookmarksopenlevel=3,
    bookmarksnumbered=true,
    hidelinks,
    colorlinks,
    linkcolor={red!50!black},
    citecolor={green!50!black},
    urlcolor={blue!80!black}
}
%\hypersetup{hidelinks}

\makeatletter
  % Recover some math symbols that were masked by eb-garamond, but do not
  % have replacement definitions.
  \DeclareSymbolFont{ntxletters}{OML}{ntxmi}{m}{it}
  \SetSymbolFont{ntxletters}{bold}{OML}{ntxmi}{b}{it}
  \re@DeclareMathSymbol{\leftharpoonup}{\mathrel}{ntxletters}{"28}
  \re@DeclareMathSymbol{\leftharpoondown}{\mathrel}{ntxletters}{"29}
  \re@DeclareMathSymbol{\rightharpoonup}{\mathrel}{ntxletters}{"2A}
  \re@DeclareMathSymbol{\rightharpoondown}{\mathrel}{ntxletters}{"2B}
  \re@DeclareMathSymbol{\triangleleft}{\mathbin}{ntxletters}{"2F}
  \re@DeclareMathSymbol{\triangleright}{\mathbin}{ntxletters}{"2E}
  \re@DeclareMathSymbol{\partial}{\mathord}{ntxletters}{"40}
  \re@DeclareMathSymbol{\flat}{\mathord}{ntxletters}{"5B}
  \re@DeclareMathSymbol{\natural}{\mathord}{ntxletters}{"5C}
  \re@DeclareMathSymbol{\star}{\mathbin}{ntxletters}{"3F}
  \re@DeclareMathSymbol{\smile}{\mathrel}{ntxletters}{"5E}
  \re@DeclareMathSymbol{\frown}{\mathrel}{ntxletters}{"5F}
  \re@DeclareMathSymbol{\sharp}{\mathord}{ntxletters}{"5D}
  \re@DeclareMathAccent{\vec}{\mathord}{ntxletters}{"7E}

  % Change font for algorithm label.
  \renewcommand\ALG@name{\sffamily\bfseries Algorithm}
\makeatother

% Use the sans-serif fonts (for which bold is properly defined) for
% section headings. Also ensure that subsubsections get a number and
% that the number is displayed (to distinguish them from paragraphs).
\allsectionsfont{\sffamily}
\setcounter{secnumdepth}{3}
\pagestyle{plain}

\makeatletter
\renewenvironment{abstract}{%
      \list{}{\advance\topsep by0.35cm\relax\small
      \leftmargin=1cm
      \labelwidth=\z@
      \listparindent=\z@
      \itemindent\listparindent
      \rightmargin\leftmargin}\item[\hskip\labelsep
                                    \textsf{\textbf{\abstractname}}]}
    {\endlist}
\makeatother

\spnewtheorem{mtheorem}{Theorem}{\sffamily\bfseries}{\itshape}
\spnewtheorem*{mproof}{Proof}{\sffamily\bfseries\itshape}{\rmfamily}

%\newcommand{\GF}{\mathrm{\textit{GF}}}
\newcommand{\GF}{GF}
\newcommand{\QR}{QR}
\newcommand{\bB}{\mathbb{B}}
\newcommand{\bF}{\mathbb{F}}
\newcommand{\bG}{\mathbb{G}}
\newcommand{\bN}{\mathbb{N}}
\newcommand{\bZ}{\mathbb{Z}}
\newcommand{\bR}{\mathbb{R}}
\newcommand{\cC}{\mathcal{C}}
\newcommand{\cG}{\mathcal{G}}
\newcommand{\neutral}{\mathbb{O}}
\newcommand{\vol}{\text{\textsf{vol}}}
\newcommand{\bitlength}{\text{\textsf{len}}}
\newcommand{\MAC}{\text{\textsf{MAC}}}
\newcommand{\MAClen}{\text{\textsf{MAClen}}}
\newcommand{\cc}{\text{\textsf{enc}}}
\newcommand{\cclen}{\text{\textsf{clen}}}
\newcommand{\Setup}{\text{\textsf{Setup}}}
\newcommand{\Eval}{\text{\textsf{Eval}}}
\newcommand{\Verify}{\text{\textsf{Verify}}}
\newcommand{\VDF}{\text{\textsf{VDF}}}
\newcommand{\VDFlen}{\text{\textsf{VDFlen}}}
\newcommand{\smod}[1]{\,\,\,(\text{mod}^{\pm} #1)}
\newcommand{\smodnospace}[1]{(\text{mod}^{\pm} #1)}
\newcommand{\sign}{\text{\textsf{sign}}}

\newcommand{\ezut}{\textsc{ezut}\xspace}
\newcommand{\xwj}{\textsc{xwj}\xspace}

\raggedbottom

\begin{document}

\title{\textsf{Double-Odd Jacobi Quartic}}

\author{Thomas Pornin}
\institute{NCC Group, \email{thomas.pornin@nccgroup.com}}

\maketitle
\noindent\makebox[\textwidth]{13 August, 2022}
%FIXME: adjust date when (re)publishing

\begin{abstract}
Double-odd curves are curves with order equal to 2 modulo 4. A prime
order group with complete formulas and a canonical encoding/decoding
process could previously be built over a double-odd curve. In this
paper, we reformulate such curves as a specific case of the Jacobi
quartic. This allows using slightly faster formulas for point
operations, as well as defining a more efficient encoding format, so
that decoding and encoding have the same cost as classic point
compression (decoding is one square root, encoding is one inversion). We
define the prime-order groups jq255e and jq255s as the application of
that modified encoding to the do255e and do255s groups. We furthermore
define an optimized signature mechanism on these groups, that offers
shorter signatures (48 bytes instead of the usual 64 bytes, for 128-bit
security) and makes signature verification faster (down to less than
83000 cycles on an Intel x86 Coffee Lake core).
\end{abstract}

% ----------------------------------------------------------------------

\section{Introduction}

Elliptic curves are a common way of defining finite groups on which the
discrete logarithm problem is believed to be hard; on top of such
groups, various protocols can be built. For secure protocol designs, a
\emph{prime-order group} is often needed. Elliptic curves with a
non-prime order, such as the well-known twisted Edwards curve
Curve25519, have led to some serious issues when used in some
protocols\cite{CreJac2019}. The order of a curve meant for cryptographic
protocols is often written as the product $hr$, with $r$ being the prime
of interest, and $h$ the \emph{cofactor}. Twisted Edwards curves allow
for very efficient computations, but their cofactor $h$ is always a
multiple of 4. A prime order group abstraction can be built on top of
some twisted Edwards curves with cofactor 4 or 8, using the
Decaf/Ristretto technique\cite{RistrettoWeb,Ham2015}. Applied to
Curve25519, the resulting group is called ristretto255 and is currently
undergoing standardization as a RFC\cite{RistrettoDraft}.

In~\cite{Por2020-5}, we investigated a different solution, focusing on
curves with order $2r$ (i.e. cofactor 2), which we called
\emph{double-odd curves}. On top of such curves, we could build prime
order group abstractions, with appropriate complete formulas for
applying the group law to group elements, as well as canonical encoding
and decoding rules. Generally speaking, the defined double-odd groups
do255e and do255s offer the usual 128-bit security level with
performance roughly similar to that of ristretto255 (general point
addition is a bit slower at 10M instead of 8M, but sequences of
doublings are faster, with a per-doubling cost down to 1M+5S for
do255e). In this paper, we improve on double-odd curves, in three
main steps:
\begin{enumerate}

    \item Through a change of variable, we reformulate double-odd curves
    as a specific case of the extended Jacobi quartic, which gives
    access to more already known formulas\cite{HisWonCarDaw2009}. We
    thus obtain complete formulas on point addition with about the same
    cost as previously (8M+3S instead of 10M), but a lower overhead for
    doublings, so that a single doubling costs only 1M+6S, and
    subsequent doublings in a sequence cost only 1M+5S each. The overall
    performance is thus improved.

    \item The new formulas are complete on the whole curve, not just on
    the subset we used in the previous double-odd curves. This allows a
    change of the encoding format, so that decoding a group element from
    its compressed format now uses only a square root operation, with no
    additional Legendre symbol computation. We thus define new groups
    called jq255e and jq255s, which are the same groups as do255e and
    do255s but with a different encoding format. Decoding and encoding
    operations are now as fast as in Curve25519; encoding is faster than
    in ristretto255.

    \item Since the encoding rules changed, we take the opportunity to
    redefine the ECDH and Schnorr signature schemes on these groups
    (e.g. to replace SHAKE with the definitely faster BLAKE2s). Using an
    idea which has already been proposed several times (including by
    Schnorr himself), we define a signature format with a reduced size:
    we can encode signatures in only 48 bytes, quite smaller than the 64
    bytes of Ed25519, while offering the same 128-bit security level.
    The reduction in size can be an important advantage, especially in
    embedded systems with severe constraints on communication channels.
    As a nice side effect, the size reduction also makes signature
    verification substantially faster; our implementation (in Rust)
    achieves signature verification in less than 83000 cycles on an
    Intel x86 Coffee Lake core (less than 93000 cycles if we include the
    cost of decoding the public key).

\end{enumerate}

In total, the two newly defined groups jq255e and jq255s achieve
performance on par with, or better than, Ed25519 and ristretto255, with
shorter signatures and faster verification, while providing the clean
prime order group abstraction with a compact, canonical and verified
encoding format, that is convenient for building secure cryptographic
protocols.

In this paper, we first recall some background on double-odd curves
(section~\ref{sec:background}), then show the conversion to a Jacobi
quartic (section~\ref{sec:quartic}). The new jq255e and jq255s groups
are formally defined in section~\ref{sec:def-groups}. Complete formulas
working over extended coordinates are detailed in
section~\ref{sec:formulas}. Section~\ref{sec:shortsig} discusses the new
signature format, and includes explicit performance benchmarks. In
appendix~\ref{sec:algspec}, a specification of the ECDH and signature
algorithms is included.

Our implementation is available as part of the \verb+crrl+ library:
\begin{center}
    \url{https://github.com/pornin/crrl}
\end{center}
This paper, as well as the original whitepaper\cite{Por2020-5}, some
graphical overview of double-odd curves, and links to various other
implementations (notably in C and in Python), are available on the
double-odd curves Web site:
\begin{center}
    \url{https://doubleodd.group/}
\end{center}

% ----------------------------------------------------------------------

\section{Background on Double-Odd Curves}\label{sec:background}

\emph{Double-odd curves} were formally defined and described
in~\cite{Por2020-5}; we recall here the principal properties of
such curves.

\paragraph{Curve Equation and Order.}
We work in a finite field $\bF_q$ of order $q$, with a characteristic
$p \geq 5$. For a field element $x \in \bF_q$, we write that $x\in\QR$
if $x$ is a square in the field, and $x\notin\QR$ otherwise. For any two
constants $(a, b) \in \bF_q\times\bF_q$ such that $a\notin\QR$ and
$a^2-4b\notin\QR$, the curve $\cC(a,b)$ is the set of points
$(x,y)\in\bF_q\times\bF_q$ such that $y^2 = x(x^2 + ax + b)$; a formal
``point-at-infinity'', denoted $\neutral$ and with no defined
coordinates $x$ or $y$, is adjoined to the curve. This curve is
double-odd, i.e. has order $2r$ for some odd integer $r$. It can be
shown that the two conditions on $a$ and $b$ imply that the curve is
well-defined and non-singular, and that any double-odd curve on $\bF_q$
can be converted to such an equation with an adequate change of
variable.

We denote $\cC(a,b)[r]$ the subgroup of $r$-torsion points, i.e. the
points $P$ on the curve such that $rP = \neutral$. A double-odd curve
contains a single point of order 2, which we denote $N$; its coordinates
are $(0, 0)$. Every point $P\in\cC(a,b)$ can be decomposed uniquely into
$P = Q + R$ with $Q$ an $r$-torsion point, and $R\in\{N,\neutral\}$.

\paragraph{Coordinates.}
For a point $P = (x,y)$, we also define the coordinates $w = y/x$ and $u
= 1/w = x/y$. For all points other than $N$ and $\neutral$, the $x$,
$y$, $w$ and $u$ coordinates are well-defined and non-zero; we formally
set $w = 0$ and $u = 0$ for both $N$ and $\neutral$. For any point $P =
(x,y,w,u)$ (with $P \neq N, \neutral$), the following properties hold:
\begin{itemize}
    \item Point $-P$ has coordinates $(x, -y, -w, -u)$.
    \item Point $P+N$ has coordinates $(b/x, -by/x^2, -w, -u)$.
    \item Exactly one of $P$ and $P+N$ is an $r$-torsion point.
    \item Coordinate $x\in\QR$ if and only if $P\in\cC(a,b)[r]$.
    \item Point $2P \in\cC(a,b)[r]$, and point $2P+N \notin\cC(a,b)[r]$.
    \item The point $-P+N$ is the only other curve point with the
    same $w$ coordinate (equivalently, the same $u$ coordinate) as $P$.
\end{itemize}

\paragraph{Prime-Order Group.}
From $\cC(a,b)$, we can define the group $\cG(a,b)$ as the quotient
group of $\cC(a,b)$ by $\{N,\neutral\}$. In other words, each element of
$\cG(a,b)$ is a pair of points $\{P,P+N\}$, where $P$ is an $r$-torsion
point and $P+N$ is not; the points $P$ and $P+N$ are called the
\emph{representants} of the group element. $\cG(a,b)$ has order $r$;
if we choose the curve such that $r$ is prime, then we obtain a prime-order
group which is convenient for building cryptographic protocols. The
original double-odd curves whitepaper\cite{Por2020-5} leveraged the
properties described above in order to efficiently implement $\cG(a,b)$:
\begin{itemize}
    \item Each group element is systematically represented by its
    non-$r$-torsion point $P+N$. Thus, the group neutral is represented
    by $N$, with defined coordinates, thereby removing the troublesome
    ``point-at-infinity''.
    \item Addition of $P+N$ and $Q+N$ can be done by using $P$ and $Q+N$,
    one being an $r$-torsion point and the other a non-$r$-torsion point,
    thus always different points (even if representing the same group
    element), which avoids the known special cases of the usual addition
    law on elliptic curves.
    \item A group element can be encoded as its $w$ coordinate; upon
    decoding, two matching curve points are rebuilt, but only one of them
    is a non-$r$-torsion point, and it can be identified by computing a
    Legendre symbol on its $x$ coordinate.
\end{itemize}
Using fractional $(x,u)$ coordinates, generic point addition can then be
computed with cost 10M, and point doubling in cost 3M+6S.

\paragraph{Isogenies.}
Some useful isogenies also exist on double-odd curves. The $\cC(a,b)$
double-odd curve is isogenous to another double-odd curve $\cC(-2a,
a^2-4b)$. Two isogenies that map between these two curves are the
following:
\begin{eqnarray*}
    \psi_1 : \cC(a, b) &\longrightarrow& \cC(-2a, a^2-4b) \\
    (x, w) &\longmapsto& \left(w^2, -\frac{(x - b/x)}{w}\right) \\
    \psi'_{1/2} : \cC(-2a, a^2-4b) &\longrightarrow& \cC(a, b) \\
    (x, w) &\longmapsto& \left(w^2/4, -\frac{(x - (a^2-4b)/x)}{2w}\right)
\end{eqnarray*}
These isogenies are in fact exactly the ones that can be obtained from
Vélu's formulas\cite{Vel1971}, and have kernel $\{N,\neutral\}$ in their
respective definition curves. Moreover, for any $P\in\cC(a,b)$, we have
$\psi'_{1/2}(\psi_1(P)) = 2P$. Using these isogenies, efficient point
doubling formulas on $\cG(a,b)$ can be defined, allowing per-doubling
overhead in long sequences of successive doublings to be as low as 1M+5S
or 2M+4S for some double-odd curves.

% ----------------------------------------------------------------------

\section{Double-Odd Curves as Jacobi Quartics}\label{sec:quartic}

For a point $P = (x,u) \in\cC(a,b)$, with $P \neq N,\neutral$, we define
an additional coordinate:
\begin{equation*}
    e = u^2 \left(x - \frac{b}{x}\right)
\end{equation*}
This value is well-defined, since curve points other than $N$ and $\neutral$
have a non-zero $x$ coordinate. Using the curve equation, we can also derive
another expression:
\begin{equation*}
    e = \frac{x^2 - b}{x^2 + ax + b}
\end{equation*}
Since $x^2 + ax + b \neq 0$ for all $x$ (otherwise, the curve would have
additional points of order 2 and would not be double-odd), this
expression is also well-defined, and can be applied to $N$ as well,
leading to $e = -1$ for that point. For reasons explained later on, we
formally define $e = 1$ for the point-at-infinity $\neutral$. Since $b
\notin\QR$, $x^2 - b$ can never be zero, and thus $e\neq 0$ for all
curve points.

Given $e$ and $u$ for a point, it is possible to recompute the $x$
coordinate by noticing that:
\begin{equation*}
    x = \frac{1}{2u^2}\left(u^2\left(x - \frac{b}{x}\right)
                            + u^2\left(x + \frac{b}{x}\right)\right)
      = \frac{1}{2u^2}\left(e + 1 - au^2\right)
\end{equation*}
We can thus use $(e,u)$ coordinates for representing points. With our
conventions, all points have such coordinates, including $N = (-1,0)$
and $\neutral = (1,0)$.

In $(e,u)$ coordinates, the curve equation becomes:
\begin{equation*}
    e^2 = (a^2-4b)u^4 - 2au^2 + 1
\end{equation*}
which is a form known as the (extended) \emph{Jacobi quartic}, first
studied by Jacobi in the 19th century\cite{Jac1829}. In the usual
formulation, the equation is denoted $Y^2 = DX^4 + 2AX^2 + 1$, with
subvariants depending on whether $D$ is a square or not. In our case,
$e$ and $u$ coordinates play the role of $Y$ and $X$, respectively, and
the $D$ constant is $a^2-4b$, which is a non-square for all double-odd
curves.

The mapping goes in both directions: a Jacobi quartic
$Y^2 = DX^4 + 2AX^2 + 1$ is turned into a curve of equation
$y^2 = x(x^2 + ax + b)$ with $a = -A$ and $b = (A^2 - D)/4$ by
setting $x = (Y + 1 + AX^2)$ and $y = x/X$. Thus, the Jacobi quartic
$Y^2 = DX^4 + 2AX^2 + 1$ is a double-odd curve if and only if
$D\notin\QR$ and $A^2-D \notin\QR$.

There exists some extensive literature on Jacobi quartics and their
formulas\cite{BilJoy2003,ChuChu1986,Jac1829,WhiWat1927}. The following
classic point addition formulas can also be derived straightforwardly
from the $(x,u)$ formulas in~\cite{Por2020-5}; for points
$P_1 = (e_1, u_1)$ and $P_2 = (e_2, u_2)$, their sum
$P_3 = (e_3, u_3) = P_1 + P_2$ can be computed as:
\begin{eqnarray*}
    e_3 &=& \frac{(1 + (a^2 - 4b) u_1^2 u_2^2)(e_1 e_2 - 2a u_1 u_2)
                  + 2(a^2 - 4b) u_1 u_2 (u_1^2 + u_2^2)}
                 {(1 - (a^2 - 4b) u_1^2 u_2^2)^2} \\
    u_3 &=& \frac{e_1 u_2 + e_2 u_1}{1 - (a^2 - 4b) u_1^2 u_2^2}
\end{eqnarray*}

These formulas are complete; they work on all curve points, including
the point of order two $N$, and the point-at-infinity $\neutral$, thanks
to the formal definitions of their coordinates as $(-1,0)$ and $(1,0)$,
respectively. We also note that when $P = (e,u)$, then $P+N = (-e,-u)$,
which further validates the choice of $(1,0)$ for $\neutral$.

% ----------------------------------------------------------------------

\section{Group Element Encoding and Decoding}\label{sec:codec}

The main virtue of the representation of curve points in the Jacobi
quartic is that the addition formulas work for all points \emph{on the
curve}. In the original definition of $\cC(a,b)$ in~\cite{Por2020-5},
the $(x,u)$ formulas were complete \emph{on the group} under the
assumption that both input points were the non-$r$-torsion representants
of their respective group elements (if the formulas were to be used on
$P$ and $-P+N$ for some point $P$, then the $x$ formula would fail).
This required a point encoding format and decoding process that reliably
return a non-$r$-torsion point. In that original description, this was
done in the following way:
\begin{enumerate}

    \item From the input field element $w$, get the two matching $x$
    coordinates by solving the equation $x^2 - (w^2 - a)x + b = 0$.
    This involves computing the discriminant $\Delta = (w^2-a)^2 - 4b$,
    then extracing its square root $\sqrt{\Delta}$. The two candidates
    are $x = ((w^2 - a) \pm \sqrt{\Delta})/2$.

    \item Choose the solution which is not a square. This requires
    computing a Legendre symbol.

\end{enumerate}
The Legendre symbol computation is normally less expensive than the
square root, but it is not negligible either, depending on the involved
architecture; for a field of about 256 bits and a large 64-bit CPU, the
cost of the Legendre symbol can be up to about 70\% that of a square
root, with optimized constant-time implementations of both operations.
Removing the need of the additional Legendre symbol would make the
decoding process noticeably faster. This can be done when using the
Jacobi quartic, though it requires changing the encoding rule, so that
the new encoding is not backward compatible with the previous one.

\paragraph{Sign.}
Let $\sign$ be an arbitrary ``sign function'' over the elements of
$\bF_q$, with the following characteristics:
\begin{itemize}

    \item For any $z \in \bF_q$, $\sign(z) = 0 \text{\ or\ } 1$.

    \item $\sign(0) = 0$.

    \item For any $z \in \bF_q$ such that $z \neq 0$,
    $\sign(-z) = 1 - \sign(z)$.

\end{itemize}
A value $z$ such that $\sign(z) = 1$ is said to be \emph{negative},
while other values are \emph{non-negative}. Any convention that fulfills
these rules is usable; in practice, when $\bF_q$ is the field of
integers modulo a prime $q$, we may use the least significant bit of $z$
when represented by an integer in the 0 to $q-1$ range. For usual
implementations of such finite fields, this sign convention can be
evaluated inexpensively, with a cost lower than that of a multiplication
in the field, i.e. much lower than that of a Legendre symbol, and
negligible with regard to the cost of a square root extraction.

\paragraph{Encoding.}
A group element represented by the point $P = (e,u)$ is encoded into a
field element $u$ with the following process:
\begin{enumerate}

    \item If $\sign(e) = 1$, then return $-u$; otherwise, return $u$.

\end{enumerate}
In other words, the group element has two representants, points
$P = (e,u)$ and $P+N = (-e,-u)$. Exactly one of these two points has
a non-negative $e$ coordinate; we return the $u$ coordinate of that
point.

\paragraph{Decoding.}
Given an input field element $u$, the decoding process works as follows:
\begin{enumerate}

    \item Compute $\Delta = (a^2-4b) u^4 - 2a u^2 + 1$.

    \item Compute $e = \sqrt{\Delta}$; if $\Delta$ has no square root,
    then the input $u$ is invalid.

    \item If $\sign(e) = 1$ then replace $e$ with $-e$.

    \item Return point $(e, u)$.

\end{enumerate}

This process simply computes $e^2$ from the curve equation, then extracts
$e$ as a square root, choosing the non-negative root, since that is the
root that the encoding process used. Note that a single square root
computation is needed, with no extra expensive operation. Moreover, the
obtained point is in affine coordinates; there is no need to merge that
root extraction with an inversion into an aggregate ``square root of a
ratio'' operation, as is commonly used in the decoding of Ed25519 public
keys\cite{EdDSArfc8032}.

Contrary to the original encoding/decoding rules from~\cite{Por2020-5},
this process may return any curve point, not necessarily the
non-$r$-torsion points only. This is not a problem if all further
operations use formulas that are complete on the curve, such as the ones
detailed thereafter. If the original $(x,u)$ formulas must be used, then
an extra Legendre symbol will be needed to select the ``right''
representant (i.e. the point whose $x$ coordinate is not a square).

\paragraph{Comparison with Decaf/Ristretto.}
The encoding/decoding process described above can be interpreted as a
reduced version of the Decaf\cite{Ham2015} process. Decaf works on a
twisted Edwards curve with cofactor 4; the cofactor is eliminated by
quotienting the curve with the subgroup of 4-torsion points. The
encoding and decoding rules then select a canonical representant through
the use of an appropriate sign convention. Ristretto\cite{RistrettoWeb}
is an adaptation of that process for twisted Edwards curves with cofactor
8, such as the well-known Curve25519.

For both Decaf and Ristretto, the decoding involves mainly a square
root computation; another square root is needed when encoding. In the
case of double-odd Jacobi quartics, we also need a square root for
decoding, but the simpler situation (cofactor is only 2) allows for a
direct encoding with only the inversion stemming from normalization
(since, in practice, points are internally represented with some sort of
fractional coordinate system). In general, inversions are more efficient
than square roots, for two reasons:
\begin{itemize}

    \item In small embedded CPUs (microcontrollers), inversion can be
    performed with an optimized binary GCD\cite{Por2020-3} or a similar
    quadratic algorithm\cite{BerYan2019} with a cost much lower than
    the modular exponentiation used at the core of a square root
    algorithm (on an ARM Cortex M0+ with the field of integers modulo
    $2^{255}-19$, inversion can be done in about 20\% of the cost of
    a square root).

    \item On large systems, the speed difference is less dramatic, but
    inversions are amenable to batching: using a trick due to
    Montgomery, a batch of $n$ field elements can be inverted with a
    single inversion in the field, and an extra $3(n-1)$
    multiplications, i.e. an asymptotic overhead of just 3
    multiplications per value to invert. No such batching optimization
    is known for square roots\footnote{As was noted in~\cite{Ham2015},
    Decaf encoding can be optimized in that way if combined with point
    doubling, i.e. encoding the double of each element instead of the
    element itself. This requires backporting that extra doubling into
    the protocol that uses Decaf, and thus is somewhat at odds with the
    goal of providing a clean prime order group abstraction.}.

\end{itemize}

Our new encoding/decoding rules provide the same performance as the
usual point compression/decompression process that can be used on any
elliptic curve.

% ----------------------------------------------------------------------

\section{The jq255e and jq255s Groups}\label{sec:def-groups}

We formally define the jq255e and jq255s groups. They are the same
groups as the do255e and do255s groups defined in~\cite{Por2020-5},
except for the encoding and decoding rules, which use the process
described above (in section~\ref{sec:codec}). These groups offer the
traditional ``128-bit'' security level, with a prime order (no
non-trivial cofactor to deal with) and canonical encoding/decoding
rules. We recall here the group parameters; see~\cite{Por2020-5}
for details on how they were chosen.

\paragraph{jq255e}
\begin{itemize}

    \item Field $\bF_q$ with $q = 2^{255} - 18651$

    \item Curve equation parameters: $(a, b) = (0, -2)$
    \begin{eqnarray*}
        y^2 &=& x(x^2 - 2) \\
        e^2 &=& 8u^4 + 1
    \end{eqnarray*}

    \item Curve order: $2r$, with $r = 2^{254} - 131528281291764213006042413802501683931$

    \item Conventional generator $G$:
    \begin{eqnarray*}
        G_e &=& -3 \\
        G_u &=& -1
    \end{eqnarray*}

\end{itemize}

This curve's specific parameters allow for the most efficient known
point doubling formulas among double-odd curves, with a per-doubling
cost of 1M+5S only when used in sequences of doublings.

This curve is also a GLV curve\cite{GalLamVan2001}, for which an
efficient non-trivial endomorphism is known. Using the notations
of~\cite{Por2020-5} (section 6.2), the function $\delta(e,u) = (e, \eta
u)$ for a given $\eta = \sqrt{-1}$ in $\bF_q$ computes in a single field
multiplication the product of a source point by a given scalar $\mu$
(which is a square root of $-1$ modulo $r$). This provides a
non-negligible speed-up for some protocols that multiply a dynamically
obtained point by a scalar, e.g. in a Diffie-Hellman key exchange.

\paragraph{jq255s}
\begin{itemize}

    \item Field $\bF_q$ with $q = 2^{255} - 3957$

    \item Curve equation parameters: $(a, b) = (-1, 1/2)$
    \begin{eqnarray*}
        y^2 &=& x(x^2 - x + 1/2) \\
        e^2 &=& -u^4 + 2u^2 + 1
    \end{eqnarray*}

    \item Curve order: $2r$, with $r = 2^{254} + 56904135270672826811114353017034461895$

    \item Conventional generator $G$:
    \begin{eqnarray*}
        G_e &=& 69296508528058375464853488337515796708 \\
            & & 37850621479164143703164723313568683024 \\
        G_u &=& 3
    \end{eqnarray*}

\end{itemize}

This curve allows per-doubling costs of 2M+4S (within sequences of
doublings). It does not otherwise have noticeable structure; e.g. its
complex multiplication discriminant is very large, as would be expected
from any randomly chosen curve. Contrary to jq255e, it does not offer an
efficient endomorphism, but it is still quite efficient as a general
purpose curve. This curve is meant as an alternative to jq255e in case
it is feared that the low CM discriminant of jq255e may lead to
exploitable weaknesses (no such attack is currently known, even though
GLV curves have been proposed and used for more than 20 years now).

\paragraph{Other parameters.}
The double-odd curves underlying jq255e and jq255s were chosen so as to
allow use of the most optimized point doubling formulas; this required
varying the field modulus, as described in~\cite{Por2020-5} (section 5).
For implementations of operations on field elements that internally use
32-bit or 64-bit limbs, all moduli $2^{255}-m$ for $m < 2^{15}$ yield
the same performance, and that kind of internal representation is what
achieves the best performance \emph{in general} on both small embedded
CPUs (e.g. ARM Cortex M0+ and M4) and large CPUs (e.g. Intel x86 with
Skylake or more recent cores). However, if $m$ is very small (i.e.
$m = 19$, the smallest $m$ such that $2^{255}-m$ is prime), then that
can lead to some minor speed-ups in representations with more limbs,
specifically 9 or 10 limbs, that themselves can better support
implementations that leverage SIMD opcodes (e.g. AVX2 or NEON), in
conjunction with specific curve point addition formulas or with
higher-level batching of operations.

With modulus $q = 2^{255}-19$, then point doubings will have cost 2M+5S
(which is still quite efficient). It is still interesting to choose
constants $a$ and $b$ such that $a$, $b$ and $a^2-4b$ are as small as
possible. Enumeration of all combinations with $|a|\leq 10$ and $|b|\leq
10$ (as plain integers) finds exactly two double-odd curves of order
$2r$ with $r$ prime: $\cC(7,8)$ and $\cC(-7,8)$ (these two curves are
isomorphic to each other, since $-1\in\QR$ in that field). Extending the
range of $a$ and $b$ to $[-20\,...+\!20]$ yields the curve $\cC(14,17)$ (and
the isomorphic $\cC(-14,17)$), which is isogenous to $\cC(7,8)$
($\psi_1$ maps from $\cC(-7,8)$ to $\cC(14, 17)$). If experiments with
$q = 2^{255}-19$ are attempted, then it is suggested to use $\cC(7,8)$.

% ----------------------------------------------------------------------

\section{Formulas}\label{sec:formulas}

For an efficient implementation, we use the extended coordinates
proposed by Hisil, Wong, Carter and Dawson\cite{HisWonCarDaw2009} (the
same coordinate system had in fact already been described by
Duquesne\cite{Duq2007}, albeit with different notations; we use here the
notations of Hisil \emph{et al}, which are more convenient). A
point $P = (e,u)$ is represented as the quadruplet $(E{:}Z{:}U{:}T)$
such that $Z\neq 0$, $e = E/Z$, $u = U/Z$ and $u^2 = T/Z$ (this
implies that $U^2 = TZ$). There are $q-1$ possible quadruplets for
a given curve point (for all possible non-zero values of $Z$).

Coordinates of $-P$ are $(E{:}Z{:}-\!U{:}T)$. Coordinates of $P+N$
are $(-E{:}Z{:}-\!U{:}T)$. Since $P$ and $P+N$ represent the same group
elements, we may freely move between $P$ and $P+N$ as is convenient.
The neutral element in the group $\cG(a,b)$ can be represented by
$N = (-Z{:}Z{:}0{:}0)$ or by $\neutral = (Z{:}Z{:}0{:}0)$, for any
$Z\neq 0$.

In the rest of this section, extended coordinates are designed as
\ezut. For implementing efficient doublings, we also make use of
Jacobian $(x,w)$ coordinates (denoted \xwj), in which a point
$P = (x,w)$ is represented as a triplet $(X{:}W{:}J)$, such that
$x = X/J^2$ and $w = W/J$; in \xwj coordinates, $N = (0{:}W{:}0)$
for any $W\neq 0$, and $\neutral = (W^2{:}W{:}0)$ for any $W\neq 0$.
Following these conventions for $N$ and $\neutral$ makes all
formulas described thereafter complete.

\subsection{Decoding}\label{sec:form-decoding}

Algorithm~\ref{alg:decode} decodes an input field element into a point
that represents an element of the group $\cG(a,b)$. In practical
situations, a binary format is needed (i.e. we encode into
\emph{bytes}); we assume here that a proper encoding of field elements
into bytes has been defined (see appendix~\ref{sec:spec-encoding}).

\begin{algorithm}[H]
    \caption{\ \ Decoding from a field element}\label{alg:decode}
    \begin{algorithmic}[1]
        \Require{$u \in \bF_q$}
        \Ensure{$P = (E{:}Z{:}U{:}T)$, or \textsc{Invalid}}
        \State{$t \leftarrow u^2$}
        \State{$\Delta \leftarrow (a^2-4b) t^2 - 2at + 1$}
        \State{$E \leftarrow \sqrt{\Delta}$
            \Comment{The $\sqrt{\,}$ operator returns \textsc{Invalid} for non-squares.}}
        \If{$E = \text{\textsc{Invalid}}$}
            \Return{\textsc{Invalid}}
        \EndIf
        \If{$\sign(E) = 1$}
            \State{$E \leftarrow -E$}
        \EndIf
        \State{$Z \leftarrow 1$}
        \State{$U \leftarrow u$}
        \State{$T \leftarrow t$}
    \end{algorithmic}
\end{algorithm}

\subsection{Encoding}\label{sec:form-encoding}

Algorithm~\ref{alg:encode} performs the reverse of
algorithm~\ref{alg:decode}: it turns a group element (represented by a
point $P$) into a field element $u$. The same $u$ is obtained for both
possible representant points of a given element of $\cG(a,b)$.

\begin{algorithm}[H]
    \caption{\ \ Encoding into a field element}\label{alg:encode}
    \begin{algorithmic}[1]
        \Require{$P = (E{:}Z{:}U{:}T)$}
        \Ensure{$u \in \bF_q$}
        \State{$K \leftarrow 1/Z$}
        \State{$u \leftarrow KU$}
        \If{$\sign(KE) = 1$}
            \State{$u \leftarrow -u$}
        \EndIf
    \end{algorithmic}
\end{algorithm}

\subsection{General Point Addition}

Given points $P_1 = (E_1{:}Z_1{:}U_1{:}T_1)$
and $P_2 = (E_2{:}Z_2{:}U_2{:}T_2)$, their sum
$P_3 = P_1 + P_2 = (E_3{:}Z_3{:}U_3{:}T_3)$ can be computed as:
\begin{eqnarray*}
    E_3 &=& (Z_1 Z_2 + (a^2-4b) T_1 T_2)(E_1 E_2 - 2a U_1 U_2)
          + 2(a^2-4b) U_1 U_2 (Z_1 T_2 + Z_2 T_1) \\
    Z_3 &=& (Z_1 Z_2 - (a^2-4b) T_1 T_2)^2 \\
    U_3 &=& (E_1 U_2 + E_2 U_1)(Z_1 Z_2 - (a^2 - 4b) T_1 T_2) \\
    T_3 &=& (E_1 U_2 + E_2 U_1)^2
\end{eqnarray*}
Algorithm~\ref{alg:add} implements these formulas.

\begin{algorithm}[H]
    \caption{\ \ Point addition (cost: 8M+3S)}\label{alg:add}
    \begin{algorithmic}[1]
        \Require{$P_1 = (E_1{:}Z_1{:}U_1{:}T_1)$ and $P_2 = (E_2{:}Z_2{:}U_2{:}T_2)$}
        \Ensure{$P_3 = P_1 + P_2 = (E_3{:}Z_3{:}U_3{:}T_3)$}
        \State{$n_1 \leftarrow E_1 E_2$}
        \State{$n_2 \leftarrow Z_1 Z_2$}
        \State{$n_3 \leftarrow U_1 U_2$}
        \State{$n_4 \leftarrow T_1 T_2$}
        \State{$n_5 \leftarrow (Z_1 + T_1)(Z_2 + T_2) - n_2 - n_4$
            \Comment{$n_5 = Z_1 T_2 + Z_2 T_1$}}
        \State{$n_6 \leftarrow (E_1 + U_1)(E_2 + U_2) - n_1 - n_3$
            \Comment{$n_6 = E_1 U_2 + E_2 U_1$}}
        \State{$n_7 \leftarrow n_2 - (a^2-4b) n_4$
            \Comment{$n_7 = Z_1 Z_2 - (a^2-4b) T_1 T_2$}}
        \State{$E_3 \leftarrow (n_2 + (a^2-4b) n_4)(n_1 - 2a n_3) + 2(a^2-4b) n_3 n_5$}
        \State{$Z_3 \leftarrow n_7^2$}
        \State{$T_3 \leftarrow n_6^2$}
        \State{$Z_3 \leftarrow ((n_6 + n_7)^2 - n_6 - n_7)/2$
            \Comment{$Z_3 = n_6 n_7$}}
    \end{algorithmic}
\end{algorithm}

The cost estimate of 8M+3S assumes that multiplication by the constants
$a^2-4b$ and $-2a$ are fast (there are three such multiplications in the
formulas above). On the jq255e group, $a^2-4b = 8$ and $-2a = 0$; on
jq255s, $a^2-4b = -1$ and $-2a = 2$. The computation of $Z_3$ can be
replaced with the simple product $n_6 n_7$; depending on the field
implementation, target system, compiler version, and the usage context,
this alternate computation may or may not provide slightly better or
slightly worse performance.

When point $P_2$ is in affine coordinates (i.e. $Z_2 = 1$), then the
computation of $n_2 = Z_1 Z_2$ vanishes, and the computation of $n_5$
simplifies into $n_5 = Z_1 T_2 + T_1$. The total cost is then slightly
lower, at 7M+3S. If both $P_1$ and $P_2$ are in affine coordinates,
then $n_2 = 1$ and $n_5 = T_1 + T_2$, for a cost of 6M+3S.

\subsection{Negation and Subtraction}

A point $P$ is negated by negating its $U$ coordinate; the other
coordinate values are unmodified. Subtraction is performed by negation
of the second operand, followed by addition. Negation in the field is
inexpensive; therefore, point subtraction has about the same cost as
point addition.

\subsection{Doubling to Jacobian (\emph{x},\emph{w})}

While point doublings can be performed correctly with the general
addition formulas, a substantially faster way involves temporarily
switching to \xwj coordinates. The following formulas compute the
point $2P = (X'{:}W'{:}J')$ from $P = (E{:}Z{:}U{:}T)$ (with three
possible choices for the computation of $W'$):
\begin{equation*}
    \begin{array}{rclclcl}
        X' &=& E^4 & & & & \\
        W' &=& 2Z^2 - 2aU^2 - E^2
           &=& Z^2 - (a^2-4b) T^2
           &=& E^2 + 2aU^2 - 2(a^2-4b) T^2 \\
        J' &=& 2EU
    \end{array}
\end{equation*}
Since we can use both $P$ and $P+N$ as representants for group elements,
we can also use the following formulas that output $2P+N = (X'{:}W'{:}J')$
from $P = (E{:}Z{:}U{:}T)$:
\begin{equation*}
    \begin{array}{rclclcl}
        X' &=& 16b U^4
           &=& 16b (TZ)^2 & & \\
        W' &=& E^2 + 2aU^2 - 2Z^2
           &=& (a^2-4b) T^2 - Z^2
           &=& 2(a^2-4b) T^2 - E^2 - 2aU^2 \\
        J' &=& 2EU
    \end{array}
\end{equation*}
It can be verified that if the source is $\neutral = (Z{:}Z{:}0{:}0)$
or $N = (-Z{:}Z{:}0{:}0)$ for some $Z \neq 0$, then these formulas
indeed yield $(Z^4{:}Z^2{:}0)$ and $(0{:}\!-\!\!Z^2{:}0)$, respectively,
i.e. correct \xwj representations of $\neutral$ and $N$.

Algorithm~\ref{alg:double-to-xwj-generic} is applicable to all
double-odd curves and computes $2P+N$ from $P$ in cost 2M+2S.

\begin{algorithm}[H]
    \caption{\ \ Doubling \ezut to \xwj (cost: 2M+2S)}\label{alg:double-to-xwj-generic}
    \begin{algorithmic}[1]
        \Require{$P = (E{:}Z{:}U{:}T)$}
        \Ensure{$P' = 2P+N = (X'{:}W'{:}J')$}
        \State{$n \leftarrow U^2$
            \Comment{$n = U^2 = TZ$}}
        \State{$X' \leftarrow 16bn^2$}
        \State{$W' \leftarrow ((a^2-4b)T + Z)(T - Z) + (a^2-4b-1)n$
            \Comment{$W' = (a^2-4b) T^2 - Z^2$}}
        \State{$J' \leftarrow 2EU$}
    \end{algorithmic}
\end{algorithm}

When working in a field $\bF_q$ where $q = 3\bmod 4$, $-1 \notin\QR$,
and therefore $4b-a^2\in\QR$, and we can use $\sqrt{4b-a^2}$ to change
the computation of $W'$ into:
\begin{equation*}
    W' = 2(\sqrt{4b-a^2})n - ((\sqrt{4b-a^2})T + Z)^2
\end{equation*}
This lowers the cost down to 1M+3S, provided that multiplications by the
constant value $\sqrt{4b-a^2}$ are fast; this is assuredly the case for
jq255s, where $4b-a^2 = 1$. Doubling to \xwj for jq255s, using that trick,
is shown in algorithm~\ref{alg:double-to-xwj-jq255s}.

\begin{algorithm}[H]
    \caption{\ \ Doubling \ezut to \xwj on jq255s (cost: 1M+3S)}\label{alg:double-to-xwj-jq255s}
    \begin{algorithmic}[1]
        \Require{$P = (E{:}Z{:}U{:}T) \in \cC(-1,1/2)$}
        \Ensure{$P' = 2P+N = (X'{:}W'{:}J')$}
        \State{$n \leftarrow U^2$
            \Comment{$n = U^2 = TZ$}}
        \State{$X' \leftarrow 8n^2$}
        \State{$W' \leftarrow 2n - (T + Z)^2$
            \Comment{$W' = - T^2 - Z^2$}}
        \State{$J' \leftarrow 2EU$}
    \end{algorithmic}
\end{algorithm}

On jq255e, and more generally on double-odd curves with $a = 0$, the
generic formulas lead to a cost 2M+2S (the requirement that both $b$ and
$a^2 - 4b$ are non-squares, combined with $a = 0$, implies that
$-4 \in\QR$, which is not possible in a field $\bF_q$ with $q = 3\bmod 4$).
However, if computing $2P$ instead of $2P+N$, then it is possible to
obtain 1M+3S formulas applicable to such curves; this is shown in
algorithm~\ref{alg:double-to-xwj-jq255e}.

\begin{algorithm}[H]
    \caption{\ \ Doubling \ezut to \xwj on $\cC(0,b)$ (e.g. jq255e) (cost: 1M+3S)}\label{alg:double-to-xwj-jq255e}
    \begin{algorithmic}[1]
        \Require{$P = (E{:}Z{:}U{:}T) \in \cC(-1,1/2)$}
        \Ensure{$P' = 2P = (X'{:}W'{:}J')$}
        \State{$n \leftarrow E^2$}
        \State{$X' \leftarrow n^2$}
        \State{$W' \leftarrow 2Z^2 - n$
            \Comment{$W' = 2Z^2 - 2aU^2 - E^2$}}
        \State{$J' \leftarrow 2EU$}
    \end{algorithmic}
\end{algorithm}

\subsection{Conversion from \xwj to \ezut}

A point $(X{:}W{:}J)$ in \xwj coordinates can be converted back to \ezut
coordinates $(E{:}Z{:}U{:}T)$ with the following:
\begin{eqnarray*}
    Z &=& W^2 \\
    T &=& J^2 \\
    U &=& JW \\
    E &=& 2X - Z + aT
\end{eqnarray*}
Since $W^2$ and $J^2$ are also computed, the product $JW$ can be done
with a squaring operation instead of a multiplication. This is illustrated
in algorithm~\ref{alg:xwj-to-ezut}.

\begin{algorithm}[H]
    \caption{\ \ Conversion from \xwj to \ezut (cost: 3S)}\label{alg:xwj-to-ezut}
    \begin{algorithmic}[1]
        \Require{$P = (X{:}W{:}J)$}
        \Ensure{$P = (E{:}Z{:}U{:}T)$}
        \State{$Z \leftarrow W^2$}
        \State{$T \leftarrow J^2$}
        \State{$U \leftarrow ((W + J)^2 - Z - T)/2$
            \Comment{$U = JW$}}
        \State{$E \leftarrow 2X - Z + aT$}
    \end{algorithmic}
\end{algorithm}

\subsection{Doublings and Sequences of Doublings}

A doubling in \ezut coordinates is done trivially by doing the doubling
to \xwj (with algorithm~\ref{alg:double-to-xwj-generic},
\ref{alg:double-to-xwj-jq255s} or \ref{alg:double-to-xwj-jq255e},
depending on the curve type), then converting back the result to \ezut
with algorithm~\ref{alg:xwj-to-ezut}. This leads to a cost of 2M+5S
generically (which matches the best generic formulas
from~\cite{HisWonCarDaw2009}), but only 1M+6S when $q = 3\bmod 4$ (a
category that includes jq255s), or when $a = 0$ (which is the case of
jq255e).

To compute a sequence of $n$ doublings, it is of course possible to
invoke the doubling process $n$ times. However, for some curves (curves
$\cC(0,b)$, such as jq255e, and also specifically curve jq255s), one can
do better by noticing that there exist faster doubling formulas for
these curves when in \xwj coordinates. This leads to the following
process:
\begin{enumerate}

    \item Compute the first doubling with output in \xwj coordinates
    (algorithm~\ref{alg:double-to-xwj-jq255s} or
    \ref{alg:double-to-xwj-jq255e}, cost 1M+3S).

    \item Compute the next $n-1$ doublings over \xwj coordinates
    (algorithm~\ref{alg:double-xwj-jq255s} or \ref{alg:double-xwj-jq255e},
    with cost 2M+4S or 1M+5S per doubling, respectively).

    \item Convert the result back to \ezut
    (algorithm~\ref{alg:xwj-to-ezut}, cost 3S).

\end{enumerate}
In total, the first doubling costs exactly as much as in the single
doubling case (1M+6S) but extra doublings are cheaper (2M+4S or 1M+5S
each). This is beneficial to algorithms that evaluate long sequences of
doublings, in particular multiplication of a point by a scalar, with
classic window optimizations, or wNAF scalar representation.

Algorithms~\ref{alg:double-xwj-jq255s} and \ref{alg:double-xwj-jq255e},
shown below, detail how fast doublings can be computed on jq255s and
jq255e in \xwj coordinates, respectively. These algorithms were already
described in the original double-odd whitepaper\cite{Por2020-5} and are
recalled here for completeness. Like all other algorithms in this paper,
they are complete.

\begin{algorithm}[H]
    \caption{\ \ Doubling on jq255s (\xwj) (cost: 2M+4S)}\label{alg:double-xwj-jq255s}
    \begin{algorithmic}[1]
        \Require{$P = (X{:}W{:}J) \in \cC(-1,1/2)$}
        \Ensure{$P' = 2P+N = (X'{:}W'{:}J')$}
        \State{$n_1 \leftarrow WJ$}
        \State{$n_2 \leftarrow n_1^2$}
        \State{$n_3 \leftarrow (W + J)^2 - 2n_1$
            \Comment{$n_3 = W^2 + J^2$}}
        \State{$X' \leftarrow 8n_2^2$}
        \State{$W' \leftarrow 2n_2 - n_3^2$}
        \State{$J' \leftarrow 2n_1(2X - n_3)$}
    \end{algorithmic}
\end{algorithm}

\begin{algorithm}[H]
    \caption{\ \ Doubling on $\cC(0,b)$ (e.g. jq255e) (\xwj) (cost: 1M+5S)}\label{alg:double-xwj-jq255e}
    \begin{algorithmic}[1]
        \Require{$P = (X{:}W{:}J) \in \cC(0,-2)$}
        \Ensure{$P' = 2P = (X'{:}W'{:}J')$}
        \State{$n_1 \leftarrow W^2$}
        \State{$n_2 \leftarrow n_1 - 2X$}
        \State{$n_3 \leftarrow n_2^2$}
        \State{$X' \leftarrow n_3^2$}
        \State{$W' \leftarrow n_3 - 2n_1^2$}
        \State{$J' \leftarrow J((W + n_2)^2 - n_1 - n_3)$
            \Comment{$J' = 2JWn_2$}}
    \end{algorithmic}
\end{algorithm}

\subsection{Equality Tests}

Two points $P_1 = (E_1{:}Z_1{:}U_1{:}T_1)$ and
$P_2 = (E_2{:}Z_2{:}U_2{:}T_2)$ can be compared with each other (as
representants of group elements) by
noticing that $P_1$ and $P_2$ represent the same group element if and
only if $\psi_1(P_1)$ and $\psi_1(P_2)$ are the same point on curve
$\cC(-2a, a^2-4b)$, since $\{\neutral, N\}$ is the kernel of the
$\psi_1$ isogeny. Moreover, for any source point $P \in \cC(a,b)$,
$\psi_1(P) \in \cC(-2a, a^2-4b)[r]$ ($\psi_1$ only outputs $r$-torsion
points), and the $u$ coordinate uniquely determines a point in the
subgroup of $r$-torsion points. Thus, we only have to compare the
$u$ coordinates of $\psi_1(P_1)$ and $\psi_1(P_2)$, which are equal
to $-U_1/E_1$ and $-U_2/E_2$, respectively.

This leads us to the following efficient test: $P_1$ and $P_2$ represent
the same element in $\cG(a,b)$ if and only if $U_1 E_2 = U_2 E_1$. This
test is complete (it also works for $N$ and $\neutral$) since the
$E$ coordinate is never zero.

A special case is when comparing a point to the group neutral: a point
$P = (E{:}Z{:}U{:}T)$ represents the neutral of $\cG(a,b)$ if and only
if $U = 0$.

\subsection{Field to Group Map}\label{sec:map-to-curve}

Mapping a field element to a group element is an important support
functionality for defining a hash-to-curve process. Two such maps were
previously specified in~\cite{Por2020-5} (sections 3.7 and 6.1.6): the
general case, applicable to all double-odd curves with $a\neq 0$, is the
Elligator2\cite{BerHamKraLan2013} map, while a special-case map is
defined for curves with $a = 0$, to which Elligator2 does not apply.
Since jq255e and jq255s use the same underlying curves as do255e and
do255s, we can use the same maps, with only an extra final conversion to
obtain the $(e,u)$ coordinates for the obtained points. The map-to-curve
processes for jq255e and jq255s are specified in
algorithms~\ref{alg:map-to-jq255e} and \ref{alg:map-to-jq255s},
respectively. In these algorithms, all square root computations return
the non-negative root.

\begin{algorithm}[H]
    \caption{\ \ Mapping a field element to $\cC(0,-2)$ (jq255e)}\label{alg:map-to-jq255e}
    \begin{algorithmic}[1]
        \Require{$f \in \bF_q$}
        \Ensure{$P = (E{:}Z{:}U{:}T) \in \cC(0, -2)$}
        \If{$f = 0$}
            \State{\Return $\neutral = (1{:}1{:}0{:}0)$}
        \EndIf
        \State{$\dot{x}_1 \leftarrow 4f^2 - 7$}
        \State{$\dot{x}_2 \leftarrow (4f^2 + 7)\sqrt{-1}$
            \Comment{Use the non-negative square root of $-1$.}}
        \State{$\ddot{x}_0 \leftarrow 4f$}
        \State{$z_1 \leftarrow 64f^7 + 176f^5 - 308f^3 - 343f$}
        \State{$z_2 \leftarrow -(64f^7 - 176f^5 - 308f^3 + 343f)\sqrt{-1}$}
        \State{$\ddot{y}_0 \leftarrow 8f^2$}
        \If{$z_1 \in \QR$}
            \State{$(\dot{x}, \ddot{x}, \dot{y}, \ddot{y}) \leftarrow (\dot{x}_1, \ddot{x}_0, \sqrt{z_1}, \ddot{y}_0)$}
        \ElsIf{$z_{n,2} \in \QR$}
            \State{$(\dot{x}, \ddot{x}, \dot{y}, \ddot{y}) \leftarrow (\dot{x}_2, \ddot{x}_0, \sqrt{z_2}, \ddot{y}_0)$}
        \Else
            \State{$(\dot{x}, \ddot{x}, \dot{y}, \ddot{y}) \leftarrow (\dot{x}_1 \dot{x}_2, \ddot{x}_0^2, \sqrt{z_1 z_2}, \ddot{y}_0^2)$}
        \EndIf
        \State{$(\dot{u}, \ddot{u}) \leftarrow (\dot{x} \ddot{y}, \ddot{x} \dot{y})$
            \Comment{$(\dot{x}/\ddot{x}, \dot{y}/\ddot{y}, \dot{u}/\ddot{u})$
            is a point on the dual curve $\cC(0,8)$.}}

        \State{$(\dot{X}, \ddot{X}) \leftarrow (-8 \dot{u}^2, \ddot{u}^2)$}
        \State{$(\dot{U}, \ddot{U}) \leftarrow (2 \dot{x} \ddot{x} \ddot{u}, \dot{u} (\dot{x}^2 - 8 \ddot{x}^2))$}
        \State{$(\dot{E}, \ddot{E}) \leftarrow (\dot{X}^2 + 2 \ddot{X}^2, \dot{X}^2 - 2 \ddot{X}^2)$}
        \State{$(E{:}Z{:}U{:}T) \leftarrow (\dot{E} \ddot{U}^2 {:} \ddot{E} \ddot{U}^2 {:} \dot{U} \ddot{U} \ddot{E} {:} \dot{U}^2 \ddot{E})$}
    \end{algorithmic}
\end{algorithm}

\begin{algorithm}[H]
    \caption{\ \ Mapping a field element to $\cC(-1,1/2)$ (jq255s)}\label{alg:map-to-jq255s}
    \begin{algorithmic}[1]
        \Require{$f \in \bF_q$}
        \Ensure{$P = (E{:}Z{:}U{:}T) \in \cC(-1, 1/2)$}
        \If{$f = \pm 1$}
            \State{\Return $\neutral = (1{:}1{:}0{:}0)$}
        \EndIf
        \State{$z_1 \leftarrow -2f^6 + 14f^4 - 14f^2 + 2$}
        \State{$z_2 \leftarrow -z_1 f^2$}
        \State{$\ddot{x} \leftarrow 1 - f^2$}
        \If{$z_1 \in \QR$}
            \State{$(\dot{x}, \dot{y}) \leftarrow (-2, \sqrt{z_1})$}
        \Else
            \State{$(\dot{x}, \dot{y}) \leftarrow (2 f^2, -\sqrt{z_2})$}
        \EndIf
        \If{$\dot{y} = 0$}
            \State{\Return $\neutral = (1{:}1{:}0{:}0)$}
        \EndIf
        \State{$(\dot{u}, \ddot{u}) \leftarrow (\dot{x} \ddot{x}, \dot{y})$
            \Comment{$(\dot{x}/\ddot{x}, \dot{y}/\ddot{x}^2, \dot{u}/\ddot{u})$
            is a point on the dual curve $\cC(2,-1)$.}}

        \State{$(\dot{X}, \ddot{X}) \leftarrow (2 \dot{u}^2, \ddot{u}^2)$}
        \State{$(\dot{U}, \ddot{U}) \leftarrow (2 \ddot{u}, \dot{x}^2 + 2 \ddot{x}^2)$}
        \State{$(n_1, n_2) = (\dot{x}(2\dot{x} - \ddot{x}), \ddot{x}(\dot{x} - \ddot{x}))$}
        \State{$(\dot{E}, \ddot{E}) \leftarrow (n_1 + n_2, n_1 - n_2)$}
        \State{$(E{:}Z{:}U{:}T) \leftarrow (\dot{E} \ddot{U}^2 {:} \ddot{E} \ddot{U}^2 {:} \dot{U} \ddot{U} \ddot{E} {:} \dot{U}^2 \ddot{E})$}
    \end{algorithmic}
\end{algorithm}

% ----------------------------------------------------------------------

\section{Short and Fast Signatures}\label{sec:shortsig}

\paragraph{Alternate Signature Encodings.}
In~\cite{Por2020-5}, digital signatures over groups do255e and do255s
were defined as classic Schnorr signatures, yielding 64-byte signatures,
the same size as standard Ed25519 signatures\cite{EdDSArfc8032}, and
with similar performance. Since jq255e and jq255s use a different
encoding format, and thus break compatibility with do255e and do255s, we
have an opportunity to define a slightly different signature scheme that
yields better performance, namely \emph{shorter encodings} (48 bytes
instead of 64), and also \emph{faster verification}.

A Schnorr signature scheme\cite{Sch1989} can be described as follows:
\begin{itemize}

    \item The private key is a secret scalar $d$ (an integer modulo the
    group order $r$). The corresponding public key is $Q = dG$, with $G$
    being the conventional base point in the group.

    \item To sign a message $m$, the following computations are performed:
    \begin{enumerate}

        \item Generate a per-signature secret scalar $k$ (uniformly among
        integers modulo $r$).

        \item Compute $R = kG$ (this is the \emph{commitment}).

        \item Compute $c = H(R \parallel Q \parallel m)$, where $H$ is
        a suitable hash function, computed over an unambiguous encoding
        of the per-signature commitment, the public key, and the message
        itself. Value $c$ is the \emph{challenge} and is interpreted as
        an integer modulo $r$.

        \item Compute $s = k + cd \bmod r$.

    \end{enumerate}
    The signature is nominally the $(R,c,s)$ triplet.

    \item Verification of the signature entails validating that the
    challenge $c$ is correct by recomputing it from $R$, $Q$ and $m$,
    and also checking that the verification equation $sG = R + cQ$
    is fulfilled.

\end{itemize}

Since the verifier recomputes the challenge $c$, that value needs not be
transmitted, so the signature can be reduced to the $(R,s)$ pair. This
is what happens with Ed25519, where $R$ and $s$ are both encoded over 32
bytes each, for a total signature size of 64 bytes. Another possible way
to encode the signature is to include $c$ instead of $R$; after all,
\emph{checking} the verification equation is equivalent to
\emph{recomputing} the point $R = sG - cQ$ and comparing it with the
received value. As long as $c$ is provided, $R$ can be omitted. The
total signature size can thus be reduced, if $c$ is defined to be
shorter than the encoding of $R$; this is what we propose here.

Using $(c,s)$ as the signature instead of $(R,s)$, and reducing the size
of $c$, are not new ideas. They were already in the original paper from
Schnorr\cite{Sch1989}. They were proposed again by Naccache and
Stern\cite{NacSte2001}, and yet again by Neven, Smart and
Warinschi\cite{NevSmaWar2009}. This last paper also includes extensive
analysis on the security of the construction depending on the
particulars of the hash function $H$ and the size of $c$; we summarize
here the facts from their analysis that are relevant to our proposal:
\begin{itemize}

    \item The security of Schnorr signatures against forgeries does not
    depend on the collision resistance of the hash function $H$.
    Thus, existing collision attacks on some hash functions are (mostly)
    irrelevant.

    \item There are known proofs of security that reduce the security
    of Schnorr signatures to the discrete logarithm, with some conditions
    on the size of the output of $H$. These proofs are not tight; they
    show that attacks, assuming ``generic'' group and hash function $H$,
    have cost at least $2^n$ with a hash function of output $2n$ bits,
    and a group of size at least $2^{3n}$ (i.e. a $3n$-bit curve, in
    our context).

    \item On the other hand, the best known attacks have cost at least
    $2^n$ if the group has size at least $2^{2n}$ and a hash
    function output of at least $n$ bits, as long as the hash function
    is not an $n$-bit ``narrow-pipe'' design (in practice, this means
    that $H$ should be a hash function that outputs $2n$ bits, then
    truncated to its first $n$ bits).

    \item Existing standard signature schemes tend to be ``in between''
    these two bounds. Notably, Ed25519 uses a group of size about
    $2^{252}$, i.e. at the low end of the range ($2n$-bit curve for $n$
    bits of security), but uses a hash function with a massive $4n$-bit
    output, that gets naturally truncated (through modular reduction) to
    the $2n$-bit size of scalars. In a sense, Ed25519 is a lot more
    conservative on the hash function security than on the curve.

\end{itemize}

The original Ed25519 paper\cite{BerDuiLanSchYan2012} quotes the papers
from Schnorr\cite{Sch1989} and Neven, Smart and
Warinschi\cite{NevSmaWar2009}, and discusses the possibility of using a
short hash output, but ultimately rejects it because ``the use of
double-size hashing helps alleviate concerns regarding hash-function
security'', without much further explanations. This is probably
explained by the historical context: that paper was written at a time
when serious weaknesses had been found in MD5 and SHA-1, and nobody
really knew whether the SHA-2 function would fall next; the SHA-3
competition was still ongoing, and there was a general unease with
regard to the security of hash functions. Another possible reason for
the choice of $(R,s)$ representation of signatures in Ed25519 is that it
is compatible with the batch verification optimized algorithms presented
in the paper, while the $(c,s)$ format is not.

\paragraph{Faster Verification.}
While we want to use shorter signatures primarily because there are many
contexts, notably in the world of embedded systems with severe
constraints on communication channels, where any gain on the signature
size is important, it turns out that using a shorter challenge $c$ also
promotes efficiency, by making signature verification faster. Let's
consider the implementation of a Schnorr signature scheme with a group
of size $r \approx 2^{2n}$, and a $2n$-bit challenge $c$. The verification
equation:
\begin{equation*}
    R = sG - cQ
\end{equation*}
entails computing a linear combination of two points, one of them being
the conventional generator point $G$. This step is commonly implemented
with Straus's algorithm\cite{Str1964} (also known as ``Shamir's
trick''), to the effect that for a group of size $2^{2n}$, about $2n$
point doublings and $4n/w$ point additions will be used (for a given
``window size'' $w$, which is typically between 4 and 6 bits, depending
on the target system, scarcity of RAM, and use of wNAF encoding of
scalars). An optimization first described by Antipa \emph{et
al}\cite{AntBroGalLamStrVan2005} consists of splitting the challenge
$c$ into two half-size ($n$ bits) integers $c_0$ and $c_1$ such that
$c = c_0/c_1 \bmod r$, and transforming the verification equation into:
\begin{equation*}
    (sc_1) G - c_1 R - c_0 Q = \neutral
\end{equation*}
Since $G$ is the fixed generator point, the point $2^n G$ can be precomputed,
and the value $sc_1$ can be written as $sc_1 = t_0 + 2^n t_1 \bmod r$,
leading to:
\begin{equation*}
    t_0 G + t_1 (2^n G) - c_1 R - c_0 Q = \neutral
\end{equation*}
which is a linear combination of \emph{four} points, but with half-width
coefficients ($t_0$, $t_1$, $c_0$ and $c_1$ all have size $n$ bits, not
$2n$ bits). Again using Straus's algorithm, this leads to the same
number of general point additions ($4n/w$) as previously, but with only
half the number of point doublings ($n$ instead of $2n$). The source
challenge $c$ must also be split into $c_0$ and $c_1$; the use of
Lagrange's lattice basis reduction algorithm, as optimized
in~\cite{Por2020-2}, makes that step fast enough (about 13k cycles on an
x86 Coffee Lake core, for a 253-bit source scalar) to make use of Antipa
\emph{et al}'s optimization worthwhile.

Now, suppose that we halve the size of the challenge $c$. The verification
equation can now be written as:
\begin{equation*}
    R = s_0 G + s_1 (2^n G) - cQ
\end{equation*}
with a split of $s$ into two $n$-bit halves: $s = s_0 + 2^n s_1$. This
is a linear combination of \emph{three} points instead of four, with
again half-width coefficients. It can thus be computed with the same
number of point doublings ($n$) but fewer general point additions
($3n/w$ instead of $4n/w$) than with Antipa \emph{et al}'s method;
moreover, the split of $s$ is immediate, and there is no need to run
Lagrange's algorithm. We also note that the point $R$ can be obtained
as an output, making this verification equation compatible with the
use of $(c,s)$ as the signature format.

\paragraph{Benchmarks.}
We implemented groups jq255s and jq255e, as well as the twisted
Edwards curve Ed25519, in the Rust language, as part of the open-source
\verb+crrl+ library available at:
\begin{center}
    \url{https://github.com/pornin/crrl}
\end{center}
The relevant library modules are called \verb+jq255s+, \verb+jq255e+ and
\verb+ed25519+, respectively. The exact specifications of the
implemented signature scheme are provided in
appendix~\ref{sec:spec-sign} (in a nutshell: the BLAKE2s hash function
is used for $H$, with an output truncated to 128 bits). Performance was
measured on an Intel i5-8259U CPU, clocked at 2.3~GHz (TurboBoost was
disabled), running Linux (Ubuntu 22.04) in 64-bit mode, and using the
Rust compiler version 1.59.0 (``stable'' channel, using LLVM 13). To
allow use of all local CPU opcodes, compilation flags
``\verb+-C target-cpu=native+'' were used. Table~\ref{tab:perf-crrl}
shows the results.

\begin{table}[H]
    \begin{center}
\begin{tabular}{|l|r|r|r|}
\hline
\textsf{\textbf{Operation}} &
\textsf{\textbf{jq255e}} &
\textsf{\textbf{jq255s}} &
\textsf{\textbf{Ed25519}} \\
\hline
    Decode point                 &   9371 &   9295 &   9491 \\
    Encode point                 &   7847 &   7874 &   7864 \\
    Point multiplication         &  72080 & 107875 & 107730 \\
    Point multiplication (base)  &  43570 &  44915 &  40456 \\
    Sign                         &  54738 &  56160 &  51581 \\
    Verify                       &  82839 &  86837 & 113983 \\
\hline
    Point size (bytes)           &     32 &     32 &     32 \\
    Scalar size (bytes)          &     32 &     32 &     32 \\
    Signature size (bytes)       &     48 &     48 &     64 \\
\hline
\end{tabular}
    \end{center}
    \caption{\label{tab:perf-crrl}Performance of the \texttt{crrl}
    implementation of jq255e, jq255s and Ed25519 on 64-bit x86
    (Intel Coffee Lake core); timing measurements in clock cycles.}
\end{table}

The ``Decode'' and ``Encode'' operations correspond to decoding a group
element from 32 bytes, and encoding it into 32 bytes, respectively.
``Point multiplication'' is multiplication of a dynamically obtained
(but already decoded) point by a full-width scalar; it is used in
the ECDH key exchange (but not in signatures). ``Point multiplication
(base)'' is the same operation but over the fixed and conventional base
point, which allows using multiple precomputed tables with affine
coordinates; this is the main operation used in key pair generation and
in signature generation. ``Sign'' is signature generation over a short
message (of length 32 bytes); apart from the multiplication of the base
point by the scalar $k$, it also includes generation of $k$ (using a
derandomized deterministic process), encoding of the point $R$,
computation of the challenge $c$, and computation of the signature
element $s$. Signature generation assumes that the encoded public key is
already available (i.e. the public key is not dynamically recomputed
from the private key for each signature). ``Verify'' applies the
signature verification, from the encoded signature as bytes, and a short
message (32 bytes); the verification public key ($Q$) is provided as an
already decoded point (for a verification primitive that receives the
public key in encoded format, the cost of the ``Decode'' operation must
be added to that of ``Verify'').

All operations except ``Verify'' are fully constant-time. This includes
the point decoding and encoding operations; timing-based side channels
leak neither information about the involved points, nor whether the
operation succeeded or not. Signature verification (``Verify'') is
assumed to work only on public data and employs non-constant-time
implementation strategies (in particular wNAF recoding of scalars).

The implementation of operations modulo $2^{255}-m$ (for integers $m =
18651$, $3957$ and $19$, respectively) uses the same code, a generic
\verb+GF255<m>+ type that receives the value of $m$ as a type parameter
(i.e. a compile-time constant). The code uses only pure Rust, with two
intrisic functions for additions and subtractions with carry
propagation, on x86 platforms, but no assembly. Thus, all three curves
benefit from the same optimizations at the field level. The jq255e and
jq255s implementations use the formulas described in this paper; the
Ed25519 implementation uses the classic extended coordinates from Hisil,
Wong, Carter and Dawson\cite{HisWonCarDaw2008}. In a sequence of
successive doublings with Ed25519, all doublings have cost 3M+4S, except
the last which has cost 4M+4S.

For point multiplication, 5-bit windows (with constant-time lookups) are
used. When working over the base point, four precomputed tables are used
(for $G$, $2^{65}G$, $2^{130}G$ and $2^{195}G$) with affine coordinates
(affine extended coordinates $(e,u,u^2)$ for jq255e and jq255s, Duif
extended coordinates for Ed25519).

It shall be noted that our performance on Ed25519 is competitive with
other optimized implementations. The eBACS/SUPERCOP benchmarking
framework\cite{eBACS} contains several implementations of Ed25519
signatures, some of which being optimized with 64-bit x86 assembly
(\verb+amd64-51-30k+ and \verb+amd64-64-24k+). The list of measurement
machines includes a system with a Coffee Lake core (called ``r24000''),
thus comparable to our test system. Signature generation is reported
with cost 48744 cycles; our own figure of 51581 cycles is less than 6\%
higher\footnote{We could probably slightly improve performance by using
more and/or larger precomputed tables for the base point; our current
implementation uses only 6144 bytes for these tables, and could easily
be doubled in size without hitting any L1 cache limit.}. For signature
verification, the SUPERCOP implementations achieve 164961 cycles,
substantially worse (+34\% time) than our code (123474 cycles, when
adding the costs of decoding the verification public key, and the
verification itself); this is in part due to the fact that our code
implements the Antipa \emph{et al}
optimization\cite{AntBroGalLamStrVan2005,Por2020-2} while SUPERCOP's
code does not. However, if we modify our code so as not to leverage that
optimization, then we can still achieve a verification cost of about
129000 cycles (about 138500 cycles with the decoding of the public key),
which is still noticeably lower than SUPERCOP's implementation cost.

Another well-known Ed25519 implementation is the Rust library
\verb+ed25519-dalek+\cite{dalek-crypto}. That library leverages AVX2
opcodes (using intrinsic functions) and some parallelism inside the
point addition formulas to compute up to four field multiplications
simultaneously. On our test system, they can perform an Ed25519
verification in about 118000 cycles\footnote{The AVX2 backend of
\texttt{ed25519-dalek} needs a ``nightly'' compiler, so for this test we
used Rust 1.65.0, from August 3rd, 2022.} (not counting the public key
decoding). That library does not use the Antipa \emph{et al}
optimization; if it did, it could presumably achieve a similar speed-up
as in our case, on the order of 15000 cycles or so, which would yield
a cost slightly above 100k cycles.

These figures vindicate our implementation strategy (i.e., trust LLVM
and don't fiddle with assembly) and mean that our measurements are a
faithful representation of the curves' ``true speed''. We see that, as
expected, point decoding and encoding have the same cost for all three
curves. Point multiplication with jq255s offers performance very similar
to that of Ed25519, but jq255e is substantially faster (cost is only 2/3
of that of Ed25519) because jq255e is a GLV curve\cite{GalLamVan2001}
and our implementation leverages the endomorphism to avoid half of the
doublings. \emph{A contrario}, when working over the base point, Ed25519
is about 10\% faster: the multiplicity of precomputed tables for the
base point means that much fewer point doublings are performed, which
favours Ed25519, whose point doubling is more expensive than that of
jq255e and jq255s, but its general point addition is somewhat faster. As
expected, the use of a half-width challenge $c$ in signatures provides a
substantial boost to signature verification; even with the decoding of
the public key, both jq255e and jq255s provide signature verification in
less than 100k cycles.

It should be remarked that in some contexts, where many signatures
should be verified at roughly the same time, then Ed25519 signatures can
use a batched verification process, that yields substantial gains: the
\verb+ed25519-dalek+ authors report a per-verification cost down to
60750 cycles on a Skylake-class Intel CPU\footnote{Batch verification
can only report whether all signatures are correct or not; if the
process fails then identifying the invalid signatures in the batch
requires more efforts. However, in most practical contexts, signatures
are almost always valid.}. This batch verification is not applicable to
our short signatures in $(c,s)$ format. To make our signatures amenable
to that process, they must use the $(R,s)$ format, which forfeits the
size advantage (signature encoding size would be 64 bytes instead of
48), while still keeping the speed gain on non-batched verifications.

From these figures, we can conclude that indeed jq255e and jq255s are
fast. They moreover provide prime order groups, a primitive adequate for
building cryptographic protocols. A prime order group can be obtained
over Curve25519 with Ristretto\cite{RistrettoWeb}; we implemented it as
well, and obtain the same performance as Ed25519 (as expected) except
for point encoding, which is slightly more expensive, and in fact costs
as much as point decoding (which is again expected, both operations
involving a square root).

% ----------------------------------------------------------------------

\section*{Acknowledgements}

We thank Kevin Henry, Giacomo Pope and Javed Samuel, who reviewed this paper.

%\newpage
\begin{thebibliography}{20}

% Ideally we would do that only for URLs, but I don't know how to do
% that in a non-painful way.
\RaggedRight

\bibitem{AmbBosFayJoyLocMur2017}
C.~Ambrose, J.~Bos, B.~Fay, M.~Joye, M.~Lochter and B.~Murray,
\emph{Differential Attacks on Deterministic Signatures},\\
\url{https://eprint.iacr.org/2017/975}

\bibitem{AntBroGalLamStrVan2005}
A.~Antipa, D.~Brown, R.~Gallant, R.~Lambert, R.~Struik and S.~Vanstone,
\emph{Accelerated Verification of ECDSA signatures},
Selected Areas in Cryptography - SAC 2005, Lecture Notes in Computer
Science, vol.~3897, pp.~307-318, 2005.

\bibitem{RistrettoWeb}
T.~Arcieri, I.~Lovecruft and H.~de Valence,
\emph{The Ristretto Group},\\
\url{https://ristretto.group/}

\bibitem{BerDuiLanSchYan2012}
D.~Bernstein, N.~Duif, T.~Lange, P.~Schwabe and B.-Y.~Yang,
\emph{High-speed high-security signatures},
Journal of Cryptographic Engineering, vol.~2, issue~2, pp.~77-89, 2012.

\bibitem{eBACS}
D.~Bernstein and T.~Lange,
\emph{eBACS: ECRYPT Benchmarking of Cryptographic Systems},\\
\url{https://bench.cr.yp.to} (accessed 4 August 2022).

\bibitem{BerYan2019}
D.~Bernstein and B.-Y.~Yang,
\emph{Fast constant-time gcd computation and modular inversion},\\
\url{https://gcd.cr.yp.to/papers.html#safegcd}

\bibitem{BerHamKraLan2013}
D.~Bernstein, M.~Hamburg, A.~Krasnova and T.~Lange,
\emph{Elligator: elliptic-curve points indistinguishable from uniform
random strings},
Proceedings of the 2013 ACM SIGSAC Conference on Computer and Communications
Security, 2013,\\
\url{https://doi.org/10.1145/2508859.2516734}

\bibitem{BilJoy2003}
O.~Billet and M.~Joye,
\emph{The Jacobi model of an elliptic curve and side-channel analysis},
AAECC-15, Lecture Notes in Computer Science, vol.~2643, pp.~34-42, 2003.

\bibitem{BriCorIcaMadRanTib2010}
É.~Brier, J.-S.~Coron, T.~Icart, D.~Madore, H.~Randriam and M.~TIbouchi,
\emph{Efficient Indifferentiable Hashing into Ordinary Elliptic Curves},
Advances in Cryptology - CRYPTO 2010, Lecture Notes in Computer Science,
vol.~6223, pp.~237-254, 2010.

\bibitem{Cha2022}
K.~Chalkias,
\emph{ed25519-unsafe-libs},\\
\url{https://github.com/MystenLabs/ed25519-unsafe-libs}

\bibitem{ChaGarNik2020}
K.~Chalkias, F.~Garillot and V.~Nikolaenko,
\emph{Taming the Many EdDSAs},
Security Standardisation Research - SSR 2020, Lecture Notes in
Computer Science, vol.~12529, pp.~67-90, 2020.

\bibitem{ChuChu1986}
D.~Chudnovsky and G.~Chudnovsky,
\emph{Sequences of numbers generated by addition
in formal groups and new primality and factorization tests},
Advances in Applied Mathematics, vol.~7, issue~4, pp.~385–434, 1986.

\bibitem{CreJac2019}
C.~Cremers and D.~Jackson,
\emph{Prime, Order Please! Revisiting Small Subgroup and Invalid Curve
Attacks on Protocols using Diffie-Hellman},
IEEE 32nd Computer Security Foundations Symposium (CSF), 2019.

\bibitem{dalek-crypto}
I.~Lovecruft and H.~de Valence,
\emph{Dalek cryptography},\\
\url{https://github.com/dalek-cryptography}

\bibitem{Duq2007}
S.~Duquesne,
\emph{Improving the arithmetic of elliptic curves in the Jacobi model},
Information Processing Letters, vol.~104, issue~3, pp.~101–105, 2007.

\bibitem{GalLamVan2001}
R.~Gallant, J.~Lambert and S.~Vanstone,
\emph{Faster Point Multiplication on Elliptic Curves with Efficient
Endomorphisms},
Advances in Cryptology - CRYPTO 2001, Lecture Notes in Computer Science,
vol.~2139, pp.~190-200, 2001.

\bibitem{Ham2015}
M.~Hamburg,
\emph{Decaf: Eliminating cofactors through point compression},
Advances in Cryptology - CRYPTO 2015, Lecture Notes in Computer Science,
vol.~9215, pp.~705-723, 2015.

\bibitem{HisWonCarDaw2008}
H.~Hisil, K.~Wong, G.~Carter and E.~Dawson,
\emph{Twisted Edwards Curves Revisited},
Advances in Cryptology - ASIACRYPT 2008, Lecture Notes in Computer Science,
vol.~5350, pp.~326-343, 2008.

\bibitem{HisWonCarDaw2009}
H.~Hisil, K.~Wong, G.~Carter and E.~Dawson,
\emph{Jacobi Quartic Curves Revisited},
Information Security and Privacy - ACISP 2009, Lecture Notes in Computer
Science, vol.~5594, pp.~452-468, 2009.

\bibitem{Jac1829}
C.~G.~J.~Jacobi,
\emph{Fundamenta nova theoriae functionum ellipticarum},
Sumtibus fratrum, 1829.

\bibitem{EdDSArfc8032}
S.~Josefsson and I.~Liusvaara,
\emph{Edwards-Curve Digital Signature Algorithm (EdDSA)},\\
\url{https://tools.ietf.org/html/rfc8032}

\bibitem{NacSte2001}
D.~Naccache and J.~Stern,
\emph{Signing on a Postcard},
Financial Cryptography - FC 2000, Lecture Notes in Computer Science,
vol.~1962, pp.~121-135, 2001.

\bibitem{NevSmaWar2009}
G.~Neven, N.~P.~Smart and B.Warinschi,
\emph{Hash function requirements for Schnorr signatures},
Journal of Mathematical Cryptology, vol.~3, issue 1, pp.~69-87, 2009.

\bibitem{Fips180}
Information Technology Laboratory,
\emph{Secure Hash Standard (SHS)},
National Institute of Standard and Technology, FIPS~180-4, 2015.

\bibitem{Fips202}
Information Technology Laboratory,
\emph{SHA-3 Standard: Permutation-Based Hash and Extendable-Output
Functions},
National Institute of Standard and Technology, FIPS~202, 2015.

\bibitem{BLAKE3}
J.~O'Connor, J.-P.~Aumasson, S.~Neves and Z.~Wilcox-O'Hearn,
\emph{BLAKE3},\\
\url{https://github.com/BLAKE3-team/BLAKE3}

\bibitem{PodSomSchLocRos2018}
D.~Poddebniak, J.~Somorovsky, S.~Schinzel, M.~Lochter and P.~Rösler,
\emph{Attacking Deterministic Signature Schemes Using Fault Attacks},
2018 IEEE European Symposium on Security and Privacy (EuroS\&P),
pp.~338-352, 2018.

\bibitem{DeterministicECDSArfc}
T.~Pornin,
\emph{Deterministic Usage of the Digital Signature Algorithm (DSA) and
Elliptic Curve Digital Signature Algorithm (ECDSA)},\\
\url{https://tools.ietf.org/html/rfc6979}

\bibitem{Por2020-2}
T.~Pornin,
\emph{Optimized Lattice Basis Reduction In Dimension 2, and Fast Schnorr
and EdDSA Signature Verification},\\
\url{https://eprint.iacr.org/2020/454}

\bibitem{Por2020-3}
T.~Pornin,
\emph{Optimized Binary GCD for Modular Inversion},\\
\url{https://eprint.iacr.org/2020/972}

\bibitem{Por2020-5}
T.~Pornin,
\emph{Double-Odd Elliptic Curves},\\
\url{https://eprint.iacr.org/2020/1558}

\bibitem{BLAKE2}
M.-J.~Saarinen and J.-P.~Aumasson,
\emph{The BLAKE2 Cryptographic Hash and Message Authentication Code (MAC)}\\
\url{https://datatracker.ietf.org/doc/html/rfc7693}

\bibitem{Sch1989}
C.~Schnorr,
\emph{Efficient identification and signatures for smart cards},
Advances in Cryptology - CRYPTO '89, Lecture Notes in Computer Science,
vol.~435, pp.~239-252, 1990.

\bibitem{Str1964}
E.~Straus,
\emph{Addition chains of vectors (problem 5125)},
American Mathematical Monthly, vol.~70, pp.~806-808, 1964.

\bibitem{RistrettoDraft}
H.~de Valence, J.~Grigg, M.~Hamburg, I.~Lovecruft, G.~Tankersly and
F.~Valsorda,
\emph{The ristretto255 and decaf448 Groups},\\
\url{https://datatracker.ietf.org/doc/html/draft-irtf-cfrg-ristretto255-decaf448-03}

\bibitem{Vel1971}
J.~Vélu,
\emph{Isogénies entre courbes elliptiques},
C.R. Acad. Sc. Paris, Série A, vol.~273, pp.~238-241, 1971.

\bibitem{WhiWat1927}
E.~Whittaker and G.~Watson,
\emph{A Course of Modern Analysis},
Cambridge University Press, 1927.

\end{thebibliography}

% ----------------------------------------------------------------------

\appendix

\section{Cryptographic Algorithm Specifications}\label{sec:algspec}

In this section, we specify strict rules for cryptographic algorithms
built on top of the jq255e and jq255s groups. We try to cover all edge
cases, as an attempt to avoid the kind of messy situation that impacts
existing deployments of Ed25519, where implementations don't always agree
with each other about the validity of some signatures, leading to issues
in some applications, especially consensus protocols\cite{ChaGarNik2020}.

In all this section, a \emph{byte} is really an \emph{octet}; it has a
numerical value in the $0$ to $255$ range, and contains eight bits,
numbered from $0$ (least significant bit) to $7$ (most significant bit).

\subsection{Encoding Rules}\label{sec:spec-encoding}

\paragraph{Field Elements.}
Elements of the finite field $\bF_q$ (with $q = 2^{255} - 18651$ for
jq255e, $2^{255} - 3957$ for jq255s) are encoded into exactly 32 bytes:
\begin{itemize}

    \item A field element $x \in \bF_q$ is considered as an integer
    in the $0$ to $q-1$ range, which is then encoded using the unsigned
    little-endian convention (least significant byte comes first).

    \item Encoding always uses 32 bytes, even if the integer could have
    fit into fewer bytes.

    \item Decoders MUST reject the source bytes, and return \textsc{Invalid},
    if any of the following conditions hold:
    \begin{itemize}

        \item The source length is not exactly 32 bytes.

        \item The integer resulting from unsigned little-endian
        interpretation of the source bytes is not in the $0$ to $q-1$
        range.

    \end{itemize}
    In particular, there is no ignored bit: even though the value of $q$
    implies that the most significant bit of the last byte is always zero,
    that bit MUST NOT be ignored by decoders. Moreover, checking that the
    most significant bit of the last byte is zero is not sufficient
    validation; for instance, an encoding of the integer $q$ itself, over
    32 bytes, MUST be rejected as \textsc{Invalid}, since it is not in the
    $0$ to $q-1$ range. There is no implicit reduction modulo $q$.

\end{itemize}

\paragraph{Scalars.}
Scalars are integers modulo $r$, with $r$ being the order of the group.
For jq255e and jq255s, $r$ is a prime integer close to $2^{254}$ (the
order of the underlying double-odd curve is $2r$). Scalars are encoded
with rules similar to those for field elements:
\begin{itemize}

    \item A scalar is encoded as an integer in the $0$ to $r-1$ range
    over exactly 32 bytes with unsigned little-endian convention.

    \item Decoders MUST verify that the received value is in the $0$
    to $r-1$ range. There is no ignored bit in the last byte. There
    is no implicit reduction modulo $r$.

\end{itemize}

\paragraph{Group Elements.}
A group element (often called ``point'', though a group element is,
strictly speaking, a set of two curve points, which are called the
\emph{representants} of the group element) is encoded by using the
process described in section~\ref{sec:codec}, and made explicit in
sections~\ref{sec:form-decoding} and \ref{sec:form-encoding}:
\begin{itemize}

    \item The sign convention ($\sign$ function) uses the least
    significant bit of the representation of the input field element as
    an integer in the $0$ to $q-1$ range; equivalently, the sign is the
    value of the bit number $0$ in the first byte of the encoding of the
    element as specified previously. Thus, $1$ is negative, but $-1$
    (represented as $q-1$) is not. $0$ is non-negative.

    \item Upon encoding, the group element representant whose $e$
    coordinate is non-negative is chosen; its $u$ coordinate is encoded
    into bytes using the rules for field elements.

    \item Upon decoding, the input bytes are decoded into a field
    element $u$, from which the corresponding coordinate $e$ is
    recomputed. If the input is valid, then there are two choices for
    the value of $e$, exactly one of which is non-negative; the
    non-negative value is chosen.

\end{itemize}

The neutral element of the group is a valid group element; its $u$
coordinate is zero (for both representants), and therefore it is encoded
as the field element $0$, which leads to exactly 32 bytes of value
\verb+0x00+.

\paragraph{Private Keys.}
A private key is a non-zero (secret) scalar. A private key is encoded with
the same rules as scalars, with one additional rule:
\begin{itemize}

    \item When decoding a private key from bytes, decoders MUST reject
    the scalar if it is a valid encoding of the value zero.

\end{itemize}

A scalar of value zero corresponds to 32 bytes of value \verb+0x00+.
Private keys are never equal to zero.

\paragraph{Public Keys.}
A public key is a non-neutral group element. If the private key is the
scalar $d$, then the public key is the group element $dG$, where $G$ is
the conventional base element in the group ($G$ has order exactly $r$).
Public keys are decoded and encoded using the rules for group elements,
with one additional rule:
\begin{itemize}

    \item When decoding a public key from bytes, decoders MUST reject
    the group element if it is a valid encoding of the neutral element.

\end{itemize}

The neutral group element encodes as 32 bytes of value \verb+0x00+.
Public keys are never equal to the neutral group element.

\subsection{Signature Specification}\label{sec:spec-sign}

\subsubsection{Pre-Hashing and Hash Function Choice}\label{sec:spec-prehash}

Apart from the used group (which is jq255e or jq255s in this
specification), the signature scheme is configured with a hash function.
There are three places where the hash function is used:
\begin{itemize}
    \item to process the message data (in ``pre-hashed'' mode);
    \item to compute the challenge $c$;
    \item to generate the per-signature secret scalar $k$.
\end{itemize}
In this specification, the BLAKE2s\cite{BLAKE2} hash function is
used for the second and third functionalities. If the message data
is pre-hashed, then it is recommended to use BLAKE2s for this step
as well.

BLAKE2s can optionally be configured (through its ``parameter block'')
with a specific output size, a secret key, a salt, a personalization
string, and a tree mode. We use none of these features. Whenever BLAKE2s
is used in this specification, it is run with its default parameters for
hashing. In particular, the hash output size is 32 bytes (256 bits);
this applies even if we later on truncate that output to its first 128
bits.

Pre-hashing the input message means that the signature is computed and
verified not on the message $m$ itself, but on $H'(m)$ for some hash
function $H'$. Pre-hashing facilitates streamed processing. If the data
is not pre-hashed (called here the ``raw mode''), then the signer must
already know the signing key pair when starting to process the message
data, and verifiers must known the public key and the signature value
when starting to process the message data. In a situation involving a
long message, signed by the sender and verified by the recipient, then
use of the raw mode implies that either the sender or the verifier must
be able to buffer the complete message. On the other hand, raw mode is
not impacted by collision attacks in $H'$, since in that case there is
no $H'$. This was the reason raw mode was preferred in the original
EdDSA\cite{BerDuiLanSchYan2012}; in the RFC that now specifies
Ed25519\cite{EdDSArfc8032}, raw mode is called ``PureEdDSA''.

Given an input message $m$ (a sequence of bytes), the \emph{prepared
message} is the byte sequence $M$ defined as follows:
\begin{itemize}

    \item In raw mode (no prehashing), $M$ is a single byte of value
    \verb+0x52+, followed by $m$ itself.

    \item In pre-hashed mode, $M$ is the concatenation of, in that
    order:
    \begin{itemize}

        \item a single byte of value \verb+0x48+;
        \item the hash function symbolic name, in ASCII;
        \item a single byte of value \verb+0x00+ (which marks the end
        of the hash function name);
        \item the hashed message $H'(m)$.

    \end{itemize}
    Hash function names use lowercase ASCII letters and digits;
    other punctuation signs are removed. Table~\ref{tab:hashnames}
    lists the currently defined names.

\end{itemize}

\begin{table}
    \begin{center}
\begin{tabular}{|l|l|c|}
\hline
\textsf{\textbf{Hash function}} &
\textsf{\textbf{Symbolic name}} &
\textsf{\textbf{Reference}} \\
\hline
    SHA-256     & \verb+sha256+    & \cite{Fips180} \\
    SHA-384     & \verb+sha384+    & \cite{Fips180} \\
    SHA-512     & \verb+sha512+    & \cite{Fips180} \\
    SHA-512/256 & \verb+sha512256+ & \cite{Fips180} \\
    SHA3-256    & \verb+sha3256+   & \cite{Fips202} \\
    SHA3-384    & \verb+sha3384+   & \cite{Fips202} \\
    SHA3-512    & \verb+sha3512+   & \cite{Fips202} \\
    BLAKE2s     & \verb+blake2s+   & \cite{BLAKE2} \\
    BLAKE2b     & \verb+blake2b+   & \cite{BLAKE2} \\
    BLAKE3      & \verb+blake3+    & \cite{BLAKE3} \\
\hline
\end{tabular}
    \end{center}
    \caption{\label{tab:hashnames}Defined hash function names (for
    use with signatures in pre-hashed mode).}
\end{table}

\subsubsection{Signature Generation}

Let $d$ be the signer's private key (a non-zero scalar), and $Q = dG$
the corresponding public key (with $G$ being the conventional base point
in the relevant group). The signature generation process on an
input message $m$ goes as follows:
\begin{enumerate}

    \item Process the input message $m$ into the prepared message $M$,
    as specified in section~\ref{sec:spec-prehash}.

    \item Generate a per-signature secret scalar $k$. The signer is free
    to use any method which ensures that each value $k$ is used with a
    single message only (for a given signing key), is secret, and is
    chosen with a distribution indistinguishable from uniform
    probability among scalars. A specific generation process is
    described later on (section~\ref{sec:sign-genk}), and is recommended
    for proper security in all implementation contexts.

    \item Compute the point $R = kG$.

    \item \label{step:siggen-challenge}Apply the BLAKE2s hash function
    on the concatenation of, in that order:
    \begin{itemize}

        \item the point $R$ (encoded into 32 bytes);

        \item the signer's public key $Q$ (encoded into 32 bytes);

        \item the prepared message $M$.

    \end{itemize}
    The challenge $c$ is then defined to be the first 16 bytes of the
    32-byte output of BLAKE2s.

    \item Compute the scalar $s = k + cd$ (interpreting $c$ as an
    integer in the $0$ to $2^{128}-1$ range, using the unsigned
    little-endian convention).

\end{enumerate}
The signature is then the concatenation of, in that order:
\begin{itemize}

    \item the challenge $c$ (16 bytes);
    
    \item the scalar $s$ (encoded into 32 bytes).

\end{itemize}
The total signature length is exactly 48 bytes.

It is theoretically possible (though very improbable) that $k = 0$; in
such a case, $R$ is the group neutral. This is considered valid.
Similarly, it is acceptable that $s = 0$ or that $c$ is a sequence of 16
bytes of value \verb+0x00+. The probability that any of these cases
occurs with a proper implementation of the signature generation is so
vanishingly small that it will not happen in practice\footnote{Or, more
accurately, when it happens, it is almost always due to some other
hardware failure.}.

\subsubsection{Signature Verification}

Signature verification uses as inputs a message $m$, a public key $Q$,
and a signature $S$, and outputs a Boolean result (\textsc{True} if the
signature is valid for the message $m$ relatively to the public key $Q$,
\textsc{False} otherwise). The process is the following:
\begin{enumerate}

    \item Process the input message $m$ into the prepared message $M$,
    as specified in section~\ref{sec:spec-prehash}.

    \item Verify that the signature $S$ has length exactly 48 bytes;
    otherwise, return \textsc{False}.

    \item Split $S$ into the challenge $c$ (first 16 bytes) and the
    encoded scalar $s_b$ (last 16 bytes), then decode $s_b$ into the
    scalar $s$. If the decoding process of $s_b$ returns
    \textsc{Invalid}, then return \textsc{False}.

    \item Compute the point $R = sG - cQ$ (interpreting $c$ as an
    integer in the $0$ to $2^{128}-1$ range using the unsigned
    little-endian convention).

    \item Using $M$, $Q$ and the just computed point $R$, recompute the
    challenge value $c'$ with the same process as in
    step~\ref{step:siggen-challenge} of the signature generation
    process.

    \item If $c = c'$, then return \textsc{True}; otherwise, return
    \textsc{False}.

\end{enumerate}

As previously pointed out, it is acceptable that $s$ or $c$ be zero,
or that $R$ be the neutral point. The comparison between $c$ and $c'$
is strict bytewise equality.

\subsubsection{Generation of the Per-Signature Scalar \emph{k}}\label{sec:sign-genk}

How the signer generates the secret scalar $k$ is invisible to
verifiers, and any method may be used as long as it fulfills the proper
security expectations of unpredictable uniform randomness. However, it
is RECOMMENDED to use the process described here, because it ensures
that an adequately safe value is obtained, even if the signer's hardware
system does not have a readily available cryptographically strong source
of randomness. Moreover, using \emph{exactly} that process makes the
implementation more easily testable against known test vectors.

The scalar $k$ is generated over the following inputs:
\begin{itemize}

    \item the signer's private key $d$;
    \item the signer's public key $Q$;
    \item the prepared message $M$;
    \item an additional seed $z$, which is a sequence of bytes of (almost)
    arbitrary length and contents.

\end{itemize}
The scalar $k$ is then computed as follows:
\begin{enumerate}

    \item The BLAKE2s hash function is evaluated over the concatenation
    of, in that order:
    \begin{itemize}

        \item the signer's private key $d$ (encoded over 32 bytes);
        \item the signer's public key $Q$ (encoded over 32 bytes);
        \item the length (in bytes) of $z$, expressed over 8 bytes
        using the unsigned little-endian convention;
        \item the value $z$ itself;
        \item the prepared message $M$.

    \end{itemize}

    \item The BLAKE2s output (32 bytes) is then interpreted as an integer
    in the $0$ to $2^{256}-1$ range, using the unsigned little-endian
    convention. This integer is then \emph{reduced} modulo $r$ to yield
    the scalar $k$.

\end{enumerate}
Note that the BLAKE2s output is reduced modulo $r$; this is \emph{not}
the usual decoding of a scalar from bytes, since it does not reject
out-of-range values.

The use of a hash function over the private key and source message is
known as \emph{derandomization}.
Ed25519\cite{BerDuiLanSchYan2012,EdDSArfc8032} uses it. It has also been
specified in general for DSA and ECDSA with arbitrary
curves\cite{DeterministicECDSArfc}. When $z$ has a fixed value (e.g. the
empty string), the process is deterministic. It is safe, though it has
been noted that fully deterministic signature generation makes the
implementation somewhat fragile with regard to some local side-channel
attacks, in particular fault
attacks\cite{AmbBosFayJoyLocMur2017,PodSomSchLocRos2018}. To strengthen
the process even in that case, a randomly generated seed $z$ can be
used; there are no specific requirements on that randomness, and even a
simple monotonic counter could be used. Even if the source of randomness
for $z$ is so flawed that it is stuck and always returns the same value,
then the derandomization process described here still makes the
signatures safe.

The process uses both the signer's private key $d$ and the public key
$Q$, even though the latter can be deterministically generated from the
former (with $Q = dG$), because that protects against a possible misuse
of the API of some possible implementations of the libraries. This was
recently reported for Ed25519\cite{Cha2022}. The gist of the issue is
that since signing generation requires knowledge of the encoded public
key (to compute the challenge), and regenerating the public key from the
private key is about as expensive as producing the signature itself,
some libraries accept that the caller provides the private and public
keys as two separate parameters, under the assumption that they will be
stored together. Then, some applications would allow attackers to
reference a private key and provide a distinct public key separately,
and that leads to key reconstruction attacks. Any signature library can
be designed to enforce regeneration of the public key from the private
key, but that implies that loading the private key incurs the
computational cost of recomputing the public key. Using the public key
as part of the derandomization process that generates $k$ is a much
cheaper way to mitigate this specific attack.

\noindent\textsf{\textbf{Warning:}} In general, producing a modular
integer by reducing a pseudorandom sequence of bytes of the same length
as the modulus yields a biased distribution, which, in the context of
Schnorr signatures, would leak information about the private key and
eventually allow adverserial reconstruction of that key. This issue is
avoided in jq255e and jq255s by the fact that for these groups, the
modulus $r$ of the ring of scalars is close enough to a power of two
that such biases are negligible (for both groups,
$|r - 2^{254}|\leq 2^{127}$). If this specification is ever adapted to
other groups, then this part of the generation of the per-signature
scalar MUST be reviewed carefully and possibly adjusted.

\subsection{ECDH Key Exchange Specification}\label{sec:spec-ecdh}

The ECDH (elliptic curve Diffie-Hellman) protocol is a key exchange
mechanism. Since it uses only one message per participant, it can also
be used for asymmetric encryption, if combined with a symmetric
encryption scheme.

\noindent\textsf{\textbf{Warning:}} In ECDH, both parties have
private/public key pairs. These key pairs are syntactically identical to
the key pairs used in Schnorr signatures, making it possible to use the
same key pair for signatures and for key exchange and asymmetric
encryption tasks. In general, this is NOT RECOMMENDED. The main flaw
with such designs is that private keys for encryption and for signatures
normally have incompatible lifecycle requirements: private keys for
encryption should have backups, since loss of the private key implies
loss of the data that was encrypted against the corresponding public
key; but private keys for signature should not have backups, since their
value lies in their exclusive control by the signer, and loss of the
private signing key does not invalidate previously issued signatures. On
a different level, possible interactions between a signing system and a
key exchange system working over the same private key are not well
studied, and some unwanted interaction might allow for faster attacks,
e.g. using the ECDH engine as a helper for making signature forgeries,
or vice versa.

In ECDH, each party runs the following process:
\begin{enumerate}

    \item Generate a new key pair $(d_1, Q_1)$. The private key $d_1$
    should be selected as a random non-zero scalar; zero is not an
    acceptable private key. The public key is computed as $Q_1 = d_1 G$.

    \item Encode $Q_1$ into the 32-byte sequence $p_1$, and send $p_1$
    to the peer.

    \item \label{step:ecdh-decode}Receive the 32-byte sequence $p_2$
    from the peer, and decoded it as the public key $Q_2$. This decoding
    process may fail (i.e. return \textsc{Invalid}) if the length of
    $p_2$ is not exactly 32 bytes, or is not a valid encoding of a group
    element, or is the valid encoding of the neutral element (the
    neutral is a valid point but not a valid public key); in such a
    case, set $Q_2$ to any other point (e.g. the base point $G$) but
    remember that the decoding failed. The Boolean flag \textsc{ok}
    is set to \textsc{True} if the decoding succeeded, \textsc{False}
    otherwise.

    \item Compute the point $k_1 Q_2$ and encode it into the 32-byte
    sequence $h$.

    \item Derive $h$ into a 32-byte symmetric key appropriate for further
    cryptographic operations by computing the BLAKE2s hash function
    over the concatenation of, in that order:
    \begin{itemize}

        \item If $p_1$ is lexicographically lower than $p_2$, then
        $p_1$ followed by $p_2$; otherwise, $p_2$ followed by $p_1$.

        \item If \textsc{ok} is \textsc{True}, then a single byte
        of value \verb+0x53+ followed by $h$; otherwise, a single
        byte of value \verb+0x46+ followed by $d_1$ (encoded into
        32 bytes).

    \end{itemize}

    \item Return the BLAKE2s output as the established symmetric key,
    along with the value of the \textsc{ok} flag.

\end{enumerate}

The use of the \textsc{ok} flag is meant to allow implementations to
hide from outsiders observing side channels (such as an error status or
computation time) whether the process worked; this can be an interesting
feature in some protocols where the involved points are hidden from the
attacker, but potentially alterable. If the decoding failed, then the
resulting key (BLAKE2s output) is unpredictable by attackers, but still
deterministic from the inputs (sending the same $p_2$ to a recipient
reusing the same $p_1$ will result in the same BLAKE2s output,
regardless of whether $p_2$ is a valid point encoding or not). Hiding
the success status of the operation is rarely needed and most
implementations are expected to treat the $p_1$ and $p_2$ values as
public data (though of course $d_1$, $d_2$ and $h$ are secret).

If the $p_1$ and $p_2$ messages were generated correctly and not altered
in transit, then both parties compute the same $h$ (since $k_1 Q_2 = k_2
Q_1$) and thus the same BLAKE2s output. The values $p_1$ and $p_2$ are
used as extra inputs to BLAKE2s so that the resulting key is bound to
the exchanged messages, which can be convenient for proving the security
of a protocol that leverages ECDH. The lexicographic ordering of $p_1$
and $p_2$ is equivalent to the numerical ordering of integers obtained
by interpreting $p_1$ and $p_2$ with the unsigned \emph{big-endian}
convention (and not the little-endian convention we used everywhere else
in this document); however, it is expected that implementations will
compare the byte values directly, e.g. with the C function
\verb+memcmp()+, if the exchanged messages $p_1$ and $p_2$ are public
data.

\subsection{Hash-to-Group Specification}\label{sec:spec-hash-to-curve}

In section~\ref{sec:map-to-curve}, maps from field elements to points
were defined, for both jq255e and jq255s. We use such maps to specify
hash-to-group operations that take as input an arbitrary sequence of
bytes, and produce points in the group. The hashing process is one-way
and produces an output distribution indistinguishable from uniformly
random selection. The general technique (due to Brier \emph{et
al}\cite{BriCorIcaMadRanTib2010}) is to use a hash function to derive
two field elements from the input, then map each field element to a
point, and finally add the two points together.

Given an input message $m$ (a sequence of bytes), we first compute the
\emph{prepared message} $M$ in exactly the same way as in
section~\ref{sec:spec-prehash}: $M$ is either a single byte of value
\verb+0x52+ followed by the raw data, or a single byte of value
\verb+0x48+ followed by the pre-hash function name (terminated by a
zero) and then $h(m)$, for some hash function $h$. Whether data should
be pre-hashed is an application choice and depends on the usage context;
as will be seen below, $M$ will be processed twice with BLAKE2s, so that
pre-hashing improves performance if $m$ is large. If pre-hashing, then
the used hash function $h$ MUST be an appropriately secure hash
function; by default, BLAKE2s is recommended.

Once $M$ is obtained, the following steps are applied:
\begin{enumerate}

    \item Compute BLAKE2s over the concatenation of a single byte of
    value \verb+0x01+ and the prepared message $M$ (in that order);
    the 32-byte output of BLAKE2s is then interpreted as an integer
    (using the unsigned little-endian convention) which is then
    \emph{reduced} modulo $q$ to yield the field element $f_1$.

    \item Map $f_1$ to the point $P_1$ using the appropriate map
    (algorithm~\ref{alg:map-to-jq255e} for jq255e,
    algorithm~\ref{alg:map-to-jq255s} for jq255s).

    \item Compute BLAKE2s over the concatenation of a single byte of
    value \verb+0x02+ and the prepared message $M$ (in that order);
    the 32-byte output of BLAKE2s is then interpreted as an integer
    (using the unsigned little-endian convention) which is then
    \emph{reduced} modulo $q$ to yield the field element $f_2$.

    \item Map $f_2$ to the point $P_2$ using the appropriate map
    (algorithm~\ref{alg:map-to-jq255e} for jq255e,
    algorithm~\ref{alg:map-to-jq255s} for jq255s).

    \item Compute $P = P_1 + P_2$ as the output of the hash-to-group
    process.

\end{enumerate}

It shall be noted that the BLAKE2s output is turned into a field element
through a modular reduction; this is distinct from \emph{decoding} a
field element, in that the latter would reject out-of-range inputs.
While, in general, reducing a binary input modulo a prime induces some
selection biases, this is not a problem for the fields used in jq255e
and jq255s because their respective orders (denoted $q$) are very close
to a power of 2, so that the resulting biases are negligible.

\end{document}
